\documentclass[11pt]{article}
\usepackage{geometry,booktabs,siunitx,tabularx,ragged2e,booktabs,caption}
\usepackage[T1]{fontenc}
\usepackage[utf8]{inputenc}
\newcolumntype{C}[1]{>{\Centering}m{#1}}
\renewcommand\tabularxcolumn[1]{C{#1}}
\usepackage[english]{babel}
%%Added by Diane
\usepackage{natbib,bookmark,hyperref,float,etoc,graphicx}
\graphicspath{{./figures/}} 
\usepackage{epstopdf}
\epstopdfsetup{outdir=./}
\usepackage{xcolor}

\hypersetup{
    colorlinks,
    linkcolor={blue},
    citecolor={blue},
    urlcolor={blue}
}

\urlstyle{same}
\bibliographystyle{econometrica} 
\usepackage{amssymb,amsfonts,amsthm,amsmath,latexsym,amstext}
\linespread{1.2}
%\setlength{\parskip}{0.5em} 
\theoremstyle{plain} 
\newtheorem{thm}{Theorem}[section]
\newtheorem{conj}{Conjecture}[section]
\newtheorem{lemma}[thm]{Lemma}
\newtheorem{prop}[thm]{Proposition}
\newtheorem{corl}[thm]{Corollary}

\theoremstyle{definition} %% needs `amsmath' package
\newtheorem{defn}{Definition}[section]
\newtheorem{exmp}{Example}[section]
\newtheorem{assump}{Assumption}[section]
\newtheorem{rmk}{Remark}[section]
\newtheorem{exer}{Exercise}[section]
\newtheorem{sol}{Solution}[section]
\newtheorem{property}{Property}[section]
\newtheorem{cond}{Condition} 
\newtheorem{example0}{\sc Example}[subsection]
\newenvironment{example}{\begin{example0}\em}{\end{example0}\par\noindent}


\begin{document}

\title{Online Appendix to ``On the Optimality of Differential Asset Taxation''}

\author{Thomas Phelan \\ Federal Reserve Bank of Cleveland}
\maketitle

This document provides a guide to the code used to produce the figures in the paper ``On the Optimality of Differential Asset Taxation''. All code is written in Python 3.6.5 and is located at \href{https://github.com/tphelanECON/diff\_cap\_tax}{https://github.com/tphelanECON/diff\_cap\_tax}. If you spot errors or have questions please email me at \href{tom.phelan@clev.frb.org}{tom.phelan@clev.frb.org}. 

\subsection*{Preliminaries}

To explain the code construction I recall some algebra from the paper. In Section 2 of the paper I defined a candidate value function 
\begin{equation}
\overline{v} = \max_{\substack{\overline{c}, x\geq 0, x\overline{c} \leq \overline{\omega} \\ -\ln \overline{c} + x^2/2<1}} \ \frac{(Sx - 1)\overline{c}}{\rho(1 + \ln \overline{c} - x^2/2)}
\label{VFxc}
\end{equation}
where $\overline{\omega} = \sqrt{\rho}\phi\sigma/(\rho \iota)$ and $S := (\Pi - \rho - \tau_I)/(\sqrt{\rho}\phi \sigma)$. In Appendix A I defined $\overline{x}$ and $\overline{\overline{x}}$ to be the solutions to $\overline{x}e^{\overline{x}^2/2} = \overline{\omega}$ and $\overline{\overline{x}}e^{\overline{\overline{x}}^2/2-1} = \overline{\omega}$, respectively. Following the explicit maximization in the proof of Proposition 2.3, the right-hand side of \eqref{VFxc} may be written as
\begin{equation}
\overline{v} = \max_{x\in [0,\overline{\overline{x}}]} \ g(S,\overline{\omega}, x)h(S,\overline{\omega}, x)
\label{VFxc2}
\end{equation}
where $g$ and $h$ are given by
\begin{equation}
\begin{aligned}
g(S,\overline{\omega}, x) & = \frac{1}{\rho}(Sx-1)e^{x^2/2}
\\ h(S,\overline{\omega}, x) & = 1_{x < \overline{x}(\overline{\omega})} + 
1_{x \geq \overline{x}(\overline{\omega})}\frac{(\overline{\omega}/x) e^{-x^2/2}}{1 + \ln(\overline{\omega}/x) - x^2/2}.
\end{aligned}
\label{gh}
\end{equation}

The stationary form of the resource constraint given in Assumption 3.1
\begin{equation}
(1-\psi)\overline{C}(S) + \psi = {\left(S\sqrt{\rho}\phi \sigma/\alpha + \rho_S/\alpha + (1/\alpha-1)\delta \right)}(1-\psi)\overline{K}(S)
\label{statRCpsi}
\end{equation}
may be simplified using the expressions in Appendix B.3, 
\begin{equation}
\begin{aligned} 
\overline{C}(S) & = \frac{\rho_D\overline{c}(S, \overline{\omega})}{\rho_D - \mu_c(S,\overline{\omega})} 
\\ \overline{K}(S) & = \frac{\rho_D\overline{c}(S, \overline{\omega})x(S, \overline{\omega})}{(\rho_D - \mu_c(S,\overline{\omega}))\sqrt{\rho}\phi \sigma} 
\end{aligned}
\label{bigCK}
\end{equation}
to yield
\begin{equation}
\begin{aligned}
& \ \ \alpha \sqrt{\rho}\phi \sigma{\left((1-\psi)\frac{\rho_D\overline{c}(S, \overline{\omega})}{\rho_D - \mu_c(S,\overline{\omega})} + \psi\right)}
\\ & = {\left(S\sqrt{\rho}\phi \sigma + \rho_S + (1 - \alpha) \delta \right)}(1-\psi)\frac{\rho_D\overline{c}(S, \overline{\omega})x(S, \overline{\omega})}{\rho_D - \mu_c(S,\overline{\omega})}
\label{statRCpsi2}
\end{aligned}
\end{equation}
which is how the root is found numerically. 


\subsection*{Code construction}

The sole class constructor for the paper is entitled captax and is located in classes.py. It contains the following methods (in the following, omegabar is $\overline{\omega}$):
\begin{itemize}
\item xbar(omegabar) and xbarbar(omegabar): $\overline{x}$ and $\overline{\overline{x}}$.
\item g(S,omegabar,x) and h(S,omegabar,x): the functions in \eqref{gh}.
\item x(S,omegabar) and c(S,omegabar): $x(S,\overline{\omega})$ and $\overline{c}(S,\overline{\omega})$ from the main text.
\item mu\_c(S,omegabar) and sig\_c(S,omegabar): mean and volatility of consumption growth, denoted $\mu_c$ and $\sigma_c$ in the main text.
\item omegahat(self,S,omegabar): the constant $\hat{\omega}$ in the collateral constraint in the benchmark decentralization.
\item S(Pi,phi): $(\Pi - \rho_S)/(\sqrt{\rho}\phi \sigma)$.
\item f(S,omegabar,phi): finds root of the equation \eqref{statRCpsi2}.
\item S\_hat(phi) and Pi\_hat(phi): $\hat{S}$ and $\hat{\Pi}$ from Section 3 of the text.
\item r(Pi,phi): interest rate in benchmark case (no private risk-sharing).
\item r\_pe(Pi,phi): interest rate with private risk-sharing.
\item taus(Pi,phi) and tausW(Pi,phi): entrepreneur and worker taxes in benchmark case (no private risk-sharing).    
\item taus\_pe(Pi,phi) and tausW\_pe(Pi,phi): entrepreneur and worker taxes with private risk-sharing.    
\item nu\_B(S,omegabar) and nu\_K(S,omegabar,Pi): the wedges from the partial equilibrium setting. 
\item check1(S,omegabar) and check2(S,omegabar): two checks corresponding to the assumptions in A.1 and A.2 of the appendix, which together ensure that the principal's value function is finite-valued. 
\end{itemize}

Figures and methods computed under the assumption of private risk-sharing have a suffix \_pe (for ``private equity''). Figures with no suffix assume that the collateral constraint is set at the most relaxed value possible, while figures with suffix 1.0 corresponds to $\overline{\iota}$ (which gives the tightest collateral constraints possible when $\phi =1$). 

\subsection*{Figures generation}

There are six scripts necessary for replication: main.py, classes.py, parameters.py, tight.py, relaxed.py, and private\_equity.py. 

\begin{itemize}
\item main.py: runs tight.py, relaxed.py and private\_equity.py.
\item parameters.py: defines the parameters used in the numerical examples.
\item classes.py: contains the class constructor used in the paper.
\item tight.py: produces figures pertaining to the case in which collateral constraints are tight ($\overline{\iota} = 1.0$) for the benchmark decentralization. In particular, this produces Figure 1 and Figure 2 from the main text. 
\item relaxed.py: produces figures pertaining to the case in which collateral constraints are relaxed ($\overline{\iota} \approx 0.5$) for the benchmark decentralization. In particular, this produces Figure 3 from the main text as well as the wedges in Figure 4 of Appendix E. 
\item private\_equity.py: produces figures for taxes and interest rates for the market structure in which there is private risk-sharing. In particular, this produces Figure 5, 6 and 7 of Appendix E. 
\end{itemize}



%\bibliographystyle{myplainnat}

%\bibliography{OpDiffCap}

\end{document}

