\documentclass[11pt]{article}
\usepackage{geometry,booktabs,siunitx,tabularx,ragged2e,booktabs,caption}
\usepackage[T1]{fontenc}
\usepackage[utf8]{inputenc}
\newcolumntype{C}[1]{>{\Centering}m{#1}}
\renewcommand\tabularxcolumn[1]{C{#1}}
\usepackage[english]{babel}
%%Added by Diane
\usepackage{natbib,bookmark,hyperref,float,etoc,graphicx}
\graphicspath{{./figures/}} 
\usepackage{epstopdf}
\epstopdfsetup{outdir=./}
\usepackage{xcolor}

\hypersetup{
    colorlinks,
    linkcolor={blue},
    citecolor={blue},
    urlcolor={blue}
}

\urlstyle{same}
\bibliographystyle{econometrica} 
\usepackage{amssymb,amsfonts,amsthm,amsmath,latexsym,amstext}
\linespread{1.2}
%\setlength{\parskip}{0.5em} 
\theoremstyle{plain} 
\newtheorem{thm}{Theorem}[section]
\newtheorem{conj}{Conjecture}[section]
\newtheorem{lemma}[thm]{Lemma}
\newtheorem{prop}[thm]{Proposition}
\newtheorem{corl}[thm]{Corollary}

\theoremstyle{definition} %% needs `amsmath' package
\newtheorem{defn}{Definition}[section]
\newtheorem{exmp}{Example}[section]
\newtheorem{assump}{Assumption}[section]
\newtheorem{rmk}{Remark}[section]
\newtheorem{exer}{Exercise}[section]
\newtheorem{sol}{Solution}[section]
\newtheorem{property}{Property}[section]
\newtheorem{cond}{Condition} 
\newtheorem{example0}{\sc Example}[subsection]
\newenvironment{example}{\begin{example0}\em}{\end{example0}\par\noindent}

\begin{document}

\title{On the Optimality of Differential Asset Taxation}

\author{Thomas Phelan\thanks{This paper previously circulated as, ``Optimal capital taxation with hidden investment''. I have benefited from discussions with V.V. Chari, Sebastian Di Tella, Narayana Kocherlakota, Dirk Krueger (the editor), Guillaume Sublet, Gustavo Ventura, Pierre Yared and especially Matthew Knowles and Sergio Ocampo. Replication files may be found at \href{https://github.com/tphelanECON/diff\_cap\_tax}{https://github.com/tphelanECON/diff\_cap\_tax}. All remaining errors are of course my own.} \\ Federal Reserve Bank of Cleveland}

\maketitle


\begin{abstract}
How should a government balance risk-sharing and redistributive concerns with the need to provide incentives for investment? Should they tax firm profits, individual savings, or simply levy lump-sum transfers? I address these questions in an environment with entrepreneurs and workers in which output is subject to privately observed shocks and firm owners can both misreport profits and abscond with a fraction of assets. When frictions in financial markets restrict private risk-sharing, the stationary efficient allocation may be implemented with linear taxes on savings and profits in a competitive equilibrium with collateral constraints. Further, the two taxes serve distinct roles and in general differ from one another. The savings tax affects consumption smoothing and may be positive or negative depending on the strength of general equilibrium effects, while the profits tax shares risk between the government and entrepreneurs, is unambiguously positive, and depends solely on the degree of frictions in financial markets.

\smallskip

\textbf{Keywords}: Optimal taxation, moral hazard, optimal contracting.

\smallskip

\textbf{JEL Codes}: D61, D63, E62. 
\end{abstract} 


\newpage

\section{Introduction} \label{Intro}

The appropriate taxation of capital income has long been a contentious issue in both policy circles and in the academic literature. The majority of studies of optimal taxation divide income into earnings (wages and salaries) or interest on risk-free savings.\footnote{For recent surveys see \cite{bastani_how_2020} and \cite{stantcheva_dynamic_2020}.} However, a recent empirical literature has documented the growing importance of private business income and has shown that such income differs from both earnings and interest in ways that may warrant a separate analysis. First, \cite{debacker_risky_2022} show that the variance of business income is more than 60 times the variance of labor income. Second, \cite{smith_capitalists_2019} show that ownership of private businesses is highly concentrated, particularly among top income groups.\footnote{E.g., Table I on page 1694 of \cite{smith_capitalists_2019} shows that the median number of owners of pass through firms with a top 1\%-0.1\% owner is 2.0.} Third, these authors also show that business income often falls significantly (on average by 82\%) upon the owner's death, suggesting that it is not solely the passive return on savings. Motivated by these facts, I characterize optimal taxes on business income and savings in an environment in which only some people can run businesses and owners are unable to perfectly diversify. I find that it is in general optimal to tax savings and profits at different rates, that these taxes are linear and constant over the lifetime of agents, and that they depend on conceptually distinct economic forces. 

I consider a perpetual youth environment in which individuals may either run their own business or work for someone else. Only some individuals have the ability to run businesses and the government cannot distinguish these people from the rest of the population. Business activity is subject to two agency frictions. First, the capital rented by each business is subject to depreciation shocks, and the entrepreneur may choose to misreport net operating income and divert some of it to their consumption. Because business income is risky, the government cannot be sure if low reported profits are due to bad luck or underreporting. I assume that diversion is socially wasteful, in the sense that each dollar diverted becomes less than one dollar of consumption. This fraction that is lost when diverted may be interpreted as a measure of the severity of the agency problem. Second, I assume that at any time, entrepreneurs may abscond with a fraction of the assets under their control, and thereafter trade only a risk-free bond. These two agency frictions are motivated by the aforementioned undiversified nature of business ownership together with the observed presence of collateral constraints.\footnote{For the empirical relevance of collateral constraints, see, e.g., \cite{cagetti_entrepreneurship_2006}.}

I characterize a particular constrained efficient allocation in which all aggregate quantities and cross-sectional distributions are constant over time. I do not restrict attention to a fixed set of policy instruments but instead allow the government to choose any allocation that respects the constraints imposed by the above agency frictions. In this environment, an allocation must specify the occupation of every agent and the amount of capital and labor delegated to each business, all as a function of the history of reported output. The ability of entrepreneurs to divert output to private consumption limits the possible degree of risk-sharing, since their consumption must depend on the (risky) output of their firm in order to induce truthful reporting to the tax authorities. The ability of entrepreneurs to abscond with a fraction of the assets under their control leads to a no-absconding constraint that limits the amount of capital that may be delegated to them.

In principle, the relationship between the history of reported output and the consumption and firm size of entrepreneurs may be arbitrarily complex. However, I show that the stationary efficient allocation can be completely characterized by the level of capital, the initial consumption of entrepreneurs and workers, and the constant mean and volatility of entrepreneurs' consumption growth. The no-absconding constraint may or may hold with equality, and it either binds at every date or never binds at all. Further, when it does not bind, a variation of the perturbation argument of \cite{rogerson_repeated_1985} reveals that the inverse Euler equation holds, just as in the setting with unobservable labor productivity. 

I then implement this allocation in a competitive equilibrium in which the market structure and taxes are chosen to respect the underlying information frictions. All agents may trade a risk-free bond in zero net supply and are subject to endogenously determined collateral constraints, the tightness of which depends on their ability to abscond with capital. The government chooses lump-sum transfers for newborns and levies taxes on savings and reported profits, where the latter is firm revenue minus wages, interest payments, depreciation and any amount that the entrepreneur misreports. The profits tax treats profits and losses symmetrically, in the sense that the tax liability of the firm owner is reduced by the value of the tax when their firm experiences losses. It therefore shares risk between the entrepreneur and the government, and partially makes up for the absence of private risk-sharing arrangements. The government is permitted to choose any taxes respecting the informational asymmetries, and so the transfers to newborns and the savings taxes can depend on an individual's occupation, provided that for these taxes and transfers the entrepreneurs have the incentive to reveal their type. 

The simple characterization of the efficient allocation is mirrored by an equally simple implementation, with all taxes linear, independent of age and history, and admitting closed-form expressions. Further, the optimal tax on profits is simply the highest level such that the owner does not wish to misreport income. In contrast, the optimal savings tax serves to affect the degree of consumption-smoothing and in principle may assume either sign. As in environments with unobservable labor productivity, the planner wishes to distort the return on savings below the complete markets value in order to reduce the future cost of providing utility. However, I show that in order for entrepreneurs to bear the efficient level of risk, the (pre-tax) borrowing costs of their business typically must also fall below the complete markets level, and that the government can indirectly affect the price of the bond through the use of intergenerational transfers and debt policy. This downward pressure on the real interest rate can exceed the wedge on the risk-free asset, and so for some parameters the entrepreneurs' savings are actually subsidized. The relative magnitude of the savings taxes on entrepreneurs and workers depends on the tightness of the collateral constraints. When these do not bind, the after-tax return on entrepreneurs' savings will always be lower than that on workers. In this case, although all taxes are independent of wealth, the model does, in a qualified sense, imply progressive savings taxes, because entrepreneurs (who are typically richer) face a lower after-tax return on their savings than workers, whose savings are subsidized. In contrast, when the collateral constraints bind, the planner wishes to backload consumption in order to relax future collateral constraints, which counteracts the desire to distort savings downwards and implies that the relative size of taxes between workers and entrepreneurs cannot always be signed.

For the benchmark case described above I suppose that there is no risk-sharing in the private sector, that the bond is in zero net supply, and that all entrepreneurs are equally productive (on average). I subsequently discuss how the results change when one relaxes each of these assumptions separately. First, when firm owners can issue equity, the profits tax becomes redundant because these contracts provide the same risk-sharing role. The interest rate in the efficient allocation is closer to the complete markets value and the taxes on the savings of entrepreneurs and workers simply adjust so that the after-tax returns are the same as in the benchmark decentralization. Second, when the government issues a risk-free bond with fixed interest rate, the analysis carries over easily provided that we allow the government 
to subsidize firm borrowing. The value of the subsidy will simply be chosen to ensure that the risk-premium associated with the business is the same as in the benchmark case. Third, when entrepreneurs differ in their expected returns, similar comments apply provided that we allow the savings taxes to depend on ex-ante productivity. If we add a population of more productive entrepreneurs to the benchmark economy, the optimal policy calls for the investment of their firms (or their cost of borowing) to be subsidized. The extension to heterogeneous entrepreneurs therefore leads to a qualified form of regressivity. However, when their collateral constraints do not bind, the savings taxes of these more productive entrepreneurs must be higher to ensure that the expected after-tax return on capital is common to all entrepreneurs. Further, despite this heterogeneity, the tax on profits continues to depend only on the severity of the agency frictions and therefore remains common to all entrepreneurs.

%\cite{erosa_optimal_2002} and consider lifecycle economies, 

\textbf{Related literature.} A vast literature, often referred to as the ``Ramsey'' approach in honor of the contribution of \cite{ramsey_contribution_1927}, has studied optimal taxation in environments in which the market structure is exogenously specified and the government has access to only linear taxes on capital and labor. As first shown by \cite{chamley_optimal_1986} and \cite{judd_redistributive_1985} and later generalized by \cite{chari_taxing_1999}, in environments with a representative agent it is typically the case that the optimal on capital income is zero in the long-run.\footnote{However, see \cite{straub_positive_2020} and \cite{chari_optimal_2020} for some qualifications and further discussion of this result.} Without attempting a comprehensive review, I note that the benchmark framework with a representative agent has been extended to include uninsurable labor income risk in various forms by \cite{aiyagari_optimal_1995}, \cite{conesa_taxing_2009} and \cite{dyrda_optimal_2022} and to include uninsurable investment (or capital income) risk by \cite{panousi_optimal_2012} and \cite{evans_optimal_2014}. In contrast with the current paper, all of these papers assume a common tax on all capital income. 

A separate branch of the literature, beginning with \cite{golosov_optimal_2003} and sometimes referred to as the \textit{New Dynamic Public Finance}, considers dynamic extensions of \cite{mirrlees_exploration_1971}. Instead of seeking the optimal policy lying within a given parametric class, this literature considers all allocations that satisfy incentive-constraints arising from informational asymmetries.\footnote{For important contributions see \cite{farhi_insurance_2013} and \cite{golosov_redistribution_2016}, and for a review of the literature see \cite{golosov_policy_2015}.} However, the majority of this literature has focused on environments in which the primary source of risk is labor productivity and capital income represents the return on saving. In this paper I follow the approach of considering all allocations that satisfy incentive-compatibility, but I also allow for the presence of multiple assets and heterogeneous returns to capital.\footnote{The notion of constrained efficiency n this paper therefore contrasts with \cite{davila_constrained_2012}, which is similar in spirit to \cite{geanakoplos_existence_1985} and assumes that markets are exogenously incomplete.} A small (but growing) number of papers share this approach. \cite{albanesi_optimal_2006} considers a two-period model with risky returns on capital and unobservable effort and studies a variety of market structures that differ in their degree of risk-sharing. \cite{shourideh_optimal_2013} considers an overlapping generations model in which agents may divert capital to consumption prior to investment, and focuses on the direct mechanism. I adopt a similar friction to \cite{shourideh_optimal_2013} but consider a continuous-time framework, which simplifies the analysis and allows me to derive clear prescriptions for taxes in a general equilibrium environment. \cite{slavik_machines_2014} study the optimality of differential capital taxation as in this paper, but focus on the distinct role of different types of capital (machines versus buildings) in production, rather than savings versus business income. 

Other recent papers that adopt a Mirrleesian approach to capital taxation in the presence of heterogeneous returns are the two-period models of \cite{gerritsen_optimal_2020} and \cite{boadway_optimal_2021}, in which such heterogeneity is exogenous, and \cite{kristjansson_optimal_2016}, in which returns are an exogenous function of effort and ability. In contrast, the current paper is motivated by the importance of business income documented by \cite{smith_capitalists_2019}, and therefore considers an environment in which entrepreneurs hire workers and rent capital and both the marginal product of capital and the cost of borrowing are endogenous. \cite{phelan_optimal_2021} also characterizes efficient allocations in an environment with business income, but omits physical capital (and hence misreporting and absconding) and instead allows the productivity of the business to depend on the history of the owner's (unobserved) effort. The study of differential asset taxation in this paper is related to a recent literature that explores the benefits of taxing capital income and wealth. Two recent contributions are \cite{boar_optimal_2022} and \cite{guvenen_use_2019}, who characterize the optimal linear taxes on capital income and wealth in environments with entrepreneurs and collateral. 

%that has studied the effects of taxes in environments with
Finally, although the primary motivation of this paper is the recent empirical evidence on business income, in my modeling I build upon an older literature within macroeconomics that has long recognized the important role of entrepreneurs. \cite{quadrini_entrepreneurship_2000} and \cite{cagetti_entrepreneurship_2006} show that allowing for the presence of entrepreneurs is essential in order to replicate stylized facts concerning the distribution of wealth. \cite{kitao_entrepreneurship_2008} considers the quantitative effects of different taxes on various forms of capital income within an environment similar to \cite{cagetti_entrepreneurship_2006}, while \cite{bruggemann_higher_2021} conducts an optimal taxation exercise within a similar environment. \cite{panousi_capital_2010} considers a continuous-time analogue of \cite{angeletos_uninsured_2007}, in which entrepreneurs face a portfolio choice among risky assets at every instant, and explores the effect of a common tax on capital income for the determination of the capital stock and interest rates. I adopt a market structure similar to \cite{panousi_capital_2010} but allow for different taxes on savings and profits and allow the restrictions on risk-sharing to arise endogenously from agency frictions.  

The outline of this paper is as follows: Section \ref{PAmodel} analyzes a principal-agent model with an exogenous interest rate and productivity of capital; Section \ref{STATsection} characterizes stationary efficient allocations in an environment in which productivity is determined by resource constraints and an aggregate production function; Section \ref{DecenGenEq} implements this allocation in a general equilibrium model with incomplete markets; Section \ref{discuss} provides intuition for the main results, discusses their robustness to various extensions (including heterogeneous entrepreneurs and private risk-sharing) and compares with the literature; Section \ref{numerical} computes a series of numerical examples; and Section \ref{conc} concludes. 


\section{Principal-agent model} \label{PAmodel}

This section characterizes the optimal risk-sharing arrangement between a risk-averse entrepreneur (she) and a risk-neutral principal (he) in an environment where the entrepreneur may operate a risky production technology, her consumption is private information, and she may abscond with a fraction of the physical assets under her control. Labor is absent from production, and both the marginal product of capital and the interest rate are exogenous. This problem will later be embedded into a macroeconomic model in which flow payoffs to the principal are determined by aggregate resource constraints for both labor and capital. 

The environment in this partial equilibrium section is a variation of that considered in \cite{di_tella_optimal_2021}.\footnote{The environment is simpler than \cite{di_tella_optimal_2021} in one respect because savings are observable, but it is not a strict simplification because I also assume the entrepreneur may abscond with a fraction of capital.} Time is continuous and extends indefinitely. Both the principal and the entrepreneur live forever and discount at the common rate $\rho > 0$. The preferences of the entrepreneur over consumption sequences are represented by 
$$
U^A(c) := \mathbb{E}{\left[\rho\int_0^{\infty} e^{-\rho t} \ln c_tdt\right]}.
$$
The entrepreneur may operate a constant-returns-to-scale technology subject to stochastic depreciation shocks. Only the entrepreneur may operate the production function and so the principal must delegate capital to the entrepreneur in order for production to take place. In addition, the entrepreneur may divert output to private consumption. When the capital delegated follows the process $k := (k_t)_{t\geq0}$ and the entrepreneur diverts an amount of output $s_tk_tdt$ per unit of time, where $s_t \in [0,\overline{s}]$ for some $\overline{s} > 0$, output $Y := (Y_t)_{t\geq0}$ net of borrowing costs $\rho + \tau_I$ evolves according to 
\begin{equation}
dY_t = (\Pi - \rho - \tau_I - s_t) k_t dt + \sigma k_t dB_t
\label{lawpartialDIVERT} 
\end{equation} 
where $(B_t)_{t\geq0}$ is distributed according to standard Brownian motion defined on a filtered probability space $(\Omega, (\mathcal{F}_t)_{t\geq0}, P)$. The constant $\Pi$ in \eqref{lawpartialDIVERT} is the marginal product of capital, $\rho$ represents the cost of borrowing, and $\tau_I$ is a tax on investment. Both $\Pi$ and $\tau_I$ are fixed exogenously in this section in order to first understand the optimal contract in partial equilibrium. In the perpetual youth economy of Section \ref{STATsection}, the marginal product of capital $\Pi$ will be determined by the aggregate production function and the number of workers in the economy and the tax $\tau_I$ will capture the extent to which capital accumulation affects the welfare of future generations, and will turn out to be negative (i.e. a subsidy).

The entrepreneur may only consume a fraction $\phi$ of the diverted amount $s_tk_tdt$ per unit of time $dt$, where $\phi \in (0,1)$ captures the deadweight loss from diversion and may be thought of as a measure of the severity of the agency problem. The specification in \eqref{lawpartialDIVERT} may be interpreted as the continuous-time limit of the discrete-time environments in which the principal delegates resources to the entrepreneur, investment is publicly observed, and the capital stock is subject to idiosyncratic shocks that are privately observed.\footnote{This friction therefore ought to be interpreted as hidden depreciation rather than hidden investment. I thank an anonymous referee for helping me understand this point.} I also assume that the entrepreneur may, at any time, take a fraction $\iota$ of the capital delegated to her and abscond, and after doing so trade only the same risk-free bond to which the principal has access. The principal is risk-neutral and so their preferences are represented by 
$$
U^P(k,c) := \mathbb{E}{\left[\int_0^{\infty} e^{-\rho t}[dY_t - c_tdt]\right]}.
$$ 
An allocation must specify the consumption of the entrepreneur, the capital delegated by the principal to the entrepreneur, and the fraction of capital the principal recommends the entrepreneur divert to private consumption, after every history of output. To be formal, first let the underlying probability space be $(C[0,\infty), (\mathcal{F}_t)_{t\geq0}, P)$, where $(\mathcal{F}_t)_{t\geq0}$ is the filtration generated by the evaluation maps and $P$ is the Wiener measure.\footnote{By ``evaluation maps'' I mean the functions defined by $x_t(\omega) := \omega(t)$ for all $\omega \in C[0,\infty)$ and $t \geq0$.} 

\begin{defn}\label{defnALLOC}
An allocation is a triple $(k, c, \tilde{s})$ of $\mathcal{F}$-adapted processes on $C[0,\infty)$ while a strategy is a single $\mathcal{F}$-adapted process $s$ on $C[0,\infty)$. 
\end{defn}

An allocation may be interpreted as a choice of the principal indicating capital delegated to the entrepreneur together with recommended amounts of consumption and output diverted. A strategy of the entrepreneur is then the choice of whether or not to follow the recommended diversion process. Since the entrepreneur's strategy is unobservable, the allocation chosen by the principal must be incentive compatible, in the sense that the entrepreneur must wish to follow the principal's recommendations after every history. When the entrepreneur varies $s$, she alters the law of motion of output and so changes the measure used to evaluate output paths. Denoting the corresponding expectation operator by $\mathbb{E}^s$, the utility from adhering to such a strategy is 
$$
U^A(k, c, \tilde{s}; s) := \mathbb{E}^s{\left[\rho\int_{0}^{\infty} e^{-\rho t}\ln (c_t + \phi s_t k_t)dt \right]}. 
$$
Associated with each allocation $(k,c,\tilde{s})$ and strategy $s$ is the utility process $(W_t)_{t\geq0}$ 
$$
W_t := \mathbb{E}^s{\left[\rho \left. \int_{t}^{\infty} e^{-\rho (t'-t)}  \ln (c_{t'} + \phi s_{t'} k_{t'})dt' \right| \mathcal{F}_t\right]} .
$$
The following lemma requires only elementary algebra. % and so the proof is omitted. 

\begin{lemma} \label{abscondLEMMAlog}
When the entrepreneur absconds with $k$ units of capital the utility from having access to a bond market with return rate $\rho$ is given by $W = \ln(\rho k)$.
\end{lemma} 

Lemma \ref{abscondLEMMAlog} implies that when an entrepreneur may abscond with a fraction $\iota$ of the capital and promised utility is given by $(W_t)_{t\geq0}$, capital $(k_t)_{t\geq0}$ is subject to the additional constraint $k_t \leq \omega e^{W_t}$ for all $t\geq0$ a.s., where $\omega := (\rho\iota)^{-1}$. 

\begin{defn}\label{ICdefn}
An allocation $(k,c,\tilde{s})$ is incentive-compatible if $U^A(k, c, \tilde{s}; \tilde{s}) \geq U^A(k, c, \tilde{s}; s)$ for all strategies $s$ and if the no-absconding constraint $k_t \leq \omega e^{W_t}$ holds for all $t \geq 0$ a.s., and the set of incentive-compatible allocations is denoted $\mathcal{A}^{IC}$. 
\end{defn}

Since $\phi < 1$, output is destroyed whenever the entrepreneur diverts assets to private consumption. To characterize efficient allocations, it is therefore without loss of generality to restrict attention to allocations with $\tilde{s} = 0$. For ease of notation I will omit reference to $\tilde{s}$ so that an allocation will be a pair $(k,c)$ rather than a triple $(k,c,0)$. When characterizing the principal's problem, I will follow an approach similar to \cite{toda_incomplete_2014} and consider a family of problems indexed by a final horizon date $T$, after which the planner must provide the entrepreneur with a constant stream of consumption and the delegated capital vanishes. The problem of the principal will then simply be defined as the limit of these finite horizon problems as the horizon tends to infinity. To this end, the set of incentive-compatible allocations defined between two dates $t$ and $T > t$ will be denoted $\mathcal{A}^{IC}_{t,T}$. 

\begin{defn}\label{Pprob}
Given a fixed final horizon date $T$, initial utility $W$, marginal product of capital $\Pi$ and investment tax $\tau_I$, the problem of the principal at date $t \in [0,T]$ is then
\begin{equation}
\begin{aligned}
V^T(W,t) = \max_{(k,c) \in \mathcal{A}^{IC}_{t,T}} & \ \mathbb{E}{\left[\int_t^Te^{-\rho (s-t)}[(\Pi - \rho - \tau_I)k_s - c_s] ds\right]} - \frac{1}{\rho}e^{-\rho (T- t)}e^{W_T}
\end{aligned}
\label{VT}
\end{equation}
since the payoff from providing utility $W$ with constant consumption is $-\rho^{-1}e^W$. The problem of the principal is then defined as the limit $V(W) := \lim_{T \rightarrow \infty} V^T(W,0)$. 
\end{defn} 

It is well known that the principal's problem is recursive in promised utility. Further, it is convenient to write utility in consumption units, defined as $u_t := e^{W_t}$, and denote the associated value function as $v(u)$. Before stating the explicit form of the value function, I explain how we can partially characterize the optimal contract by exploiting homogeneity and perturbation arguments.

First, note that if $\mathcal{A}(u)$ denotes the set of capital and consumption processes that satisfy incentive compatibility and promise-keeping given utility $u$, then substitution into the law of motion \eqref{LoM} implies that for any scalar $\lambda > 0$, $(k,c) \in \mathcal{A}(u)$ if and only if $(ke^{\lambda},ce^{\lambda}) \in \mathcal{A}(\lambda u)$. Since the principal's objective is homogeneous of degree one in capital and consumption, this implies that the value and policy functions are multiples of $u$, at least if the value function is finite. In this case the problem of the principal reduces to simply choosing two scalars, $\overline{k}$ and $\overline{c}$, indicating capital and consumption per unit of utility. 

Second, a perturbation argument may be employed to derive the optimal degree of intertemporal distortions. As mentioned in the introduction, a large literature has extended the static model of \cite{mirrlees_exploration_1971} to dynamic environments with privately-observed labor productivity shocks. An important observation in this literature, first established by \cite{rogerson_repeated_1985} in a principal-agent setting and later by \cite{golosov_optimal_2003} in a dynamic Mirrleesian setting, is that intertemporal distortions are characterized by an inverse Euler equation. This result rests on the insight that if an allocation is efficient, it cannot be possible to perturb it in such a way that the payoff to the principal is increased and incentive-compatibility is preserved. A similar argument is applicable here, provided that the perturbation alters delegated capital as well as consumption. The fact that preferences are logarithmic implies that $(k,c)$ is incentive compatible if and only if $(\eta k,\eta c)$ is incentive compatible for any deterministic sequence $\eta$, and this provides us with a convenient class of perturbations. Suppose that $(k,c)$ is the efficient allocation and for any real scalar $z$ and positive $t_0 < t_1$ and $dt$ define %$(\eta_t(z))_{t\geq0}$ by 
$$
\eta_t(z) = \left\{
\begin{array}{ll}
       e^z & \textnormal{if} \ t \in [t_0,t_0+dt] \\
       e^{-ze^{\rho(t_1-t_0)}}  & \textnormal{if} \ t \in [t_1,t_1+dt].
\end{array}
\right.
$$ %if the no-absconding constraint is slack, 
The change in utility is approximately $\rho [e^{-\rho t_0} - e^{-\rho t_1} e^{\rho(t_1-t_0)}]zdt = 0$ and so $\eta_t$ preserves both promise-keeping and incentive-compatibility. By the assumed efficiency of the allocation, profits must be maximized at $z=0$. The change in profits from this perturbation is approximately $(\Pi - \rho - \tau_I)[e^{-\rho t_0}k_{t_0}e^z + e^{-\rho t_1}\mathbb{E}[k_{t_1}]e^{-ze^{\rho(t_1-t_0)}}]dt$, and so differentiating with respect to $z$ and evaluating at zero gives $k_{t_0} = \mathbb{E}[k_{t_1}]$. Since $k/c$ is constant by the above homogeneity argument, this gives $c_{t_0} = \mathbb{E}{\left[c_{t_1}\right]}$, which is the inverse Euler equation. 

However, there are two potential problems with the above homogeneity and perturbation arguments. First, the principal's problem may fail to be finite-valued. For example, if $\phi = 0$ and $(\Pi - \rho - \tau_I)\omega > 1$, then the payoff from $(c_t, k_t) = (u_t, \omega u_t)$ is increasing and convex in promised utility, and so the principal could obtain arbitrarily high profits by offering the entrepreneur a lottery over contracts.\footnote{Indeed, \cite{di_tella_optimal_2021} show that without the no-absconding constraint, the value function is not finite-valued for any $\phi \in [0,1]$ when utility is logarithmic, and so the no-absconding constraint is crucial even if it does not hold with equality.} Second, even if the value function were finite, if the no-absconding constraint holds with equality then the above perturbed allocation will not be incentive-compatible for $z > 0$. The main content behind Propositions \ref{existence} and \ref{suffLEM} below is that both of these potential problems do not arise provided that the excess return on capital is sufficiently small. 
%\footnote{See, e.g., equation (12) and footnote 11 of \cite{di_tella_optimal_2021}.}

Standard arguments, reviewed in Appendix \ref{agency_IC}, ensure that incentive-compatibility is equivalent to the requirement that utility follow a diffusion process with volatility weakly exceeding the benefit of diverting output to consumption. Corollary \ref{corl_IC} shows that we may assume without loss that utility in consumption units, $u_t$, follows the process
\begin{equation} %dW_t = \rho{\left(W_t - \ln c_t\right)}dt + \rho \sigma \phi k_tc_t^{-1}dB_t. 
du_t = \rho{\left( -\ln \overline{c}_t + x_t^2/2\right)}u_tdt + \sqrt{\rho} x_tu_tdB_t
\label{LoM}
\end{equation} 
where I changed variables to $x_t := \sqrt{\rho} \phi \sigma k_t/c_t$ and $\overline{c}_t= c_t/u_t$ for convenience. To understand the intuition behind the above law of motion, note that the elasticity of utility with respect to output shocks is $\rho \phi k_t/c_t$, which is simply the product of the marginal utility of consumption with the amount of additional consumption per unit of diversion. I first consider the problem of a principal who must choose consumption growth to be smaller than the rate of discount, 
\begin{equation}
\overline{v} = \sup_{\substack{\overline{c},x\geq 0, x\overline{c} \leq \overline{\omega} \\ -\ln \overline{c} + x^2/2<1}} \ \frac{(Sx - 1)\overline{c}}{\rho(1 + \ln \overline{c} - x^2/2)}
\label{VFxc}
\end{equation}
where $\overline{\omega} = \sqrt{\rho}\phi\sigma/(\rho \iota)$ and
\begin{equation}
\begin{aligned}
S & := \frac{\Pi - \rho - \tau_I}{\sqrt{\rho}\phi \sigma}.
\end{aligned}
\label{newQ}
\end{equation} 
Writing the problem in this manner is convenient because it illustrates that the choices of the principal are summarized by the parameters $S$ and $\overline{\omega}$. In what follows I will write the efficient levels of consumption and consumption volatility as $\overline{c}(S, \overline{\omega})$ and $\sqrt{\rho} x(S, \overline{\omega})$, respectively, wherever these quantities are well-defined. Note that the maximand on the right-hand side of \eqref{VFxc} is the flow payoff to the principal divided by the difference between the discount rate and the mean growth of consumption, and so the resulting expression is simply an instance of the Gordon growth formula. However, this maximization is subtle because the maximand is not concave in the choice variables. To avoid an arbitrage opportunity, this maximand must obviously be negative at all points in the constraint set. The following shows that this is assured, and that the principal's problem is finite-valued, provided that the excess return on capital is sufficiently small. 

%The following shows that this is the only assumption necessary to ensure that the value function of the principal is finite-valued. 
\begin{prop} \label{existence}
For any $\overline{\omega}$, there exists $\overline{S}(\overline{\omega})$ such that the principal's problem is finite-valued if $S < \overline{S}(\overline{\omega})$, in which case it is $v(u) = \overline{v}u$ for all $u > 0$, where $\overline{v}$ is given in \eqref{VFxc}. Further, $x(S,\overline{\omega})$ is increasing in $S$ on the interval $[0, \overline{S}(\overline{\omega}))$. 
\end{prop}

The proof of Proposition \ref{existence} is contained in Appendix \ref{agencyAPP}. Proposition \ref{existence} confirms the validity of the above homogeneity argument, and implies that the principal's problem reduces to making just two choices wherever it is finite-valued. Consumption therefore admits the representation $dc_t/c_t = \mu_cdt + \sigma_cdB_t$ for some constants $\mu_c$ and $\sigma_c$, and the no-absconding constraint either never holds with equality or holds with equality after every history. When it holds as a strict inequality, the above perturbation argument is applicable and consumption is a martingale and satisfies $\overline{c} = e^{x^2/2}$. For this choice of consumption, if $S < 1/2$ then the maximand in \eqref{VFxc} is $\rho^{-1}(Sx-1)e^{x^2/2}$ and has local maximum
\begin{equation}
x_{\textnormal{loc}}(S) := \frac{1 - \sqrt{1 - 4S^2}}{2S}.
\label{kcx}
\end{equation}
The following is equivalent to assuming that the no-absconding constraint holds as a strict inequality in the optimal contract. 

\begin{assump} \label{locfeas}%simply the solution to the quadratic $0 = Sx^2 - x + S$, or
The parameter $S$ lies in $[0, 1/2]$ and $x_{\textnormal{loc}}(S) = x(S,\overline{\omega})$. 
\end{assump}

Similar to Proposition \ref{existence}, the following shows that the above assumption is satisfied for all sufficiently small excess return on capital. 

\begin{prop}\label{suffLEM} 
The no-absconding constraint will hold as a strict inequality if Assumption \ref{locfeas} holds. Further, for each $\overline{\omega} > 0$ there exists $S_{\textnormal{loc}}(\overline{\omega})$ such that Assumption \ref{locfeas} holds if and only if $S \in [0, S_{\textnormal{loc}}(\overline{\omega})]$. 
\end{prop} 

The proof of Proposition \ref{suffLEM} is contained in Appendix \ref{agencyAPP}. Note that when the no-absconding constraint does not hold with equality, consumption is a martingale and has zero drift. Using this and the law of motion \eqref{LoM}, the drift in consumption is 
\begin{equation}
\mu_c(S,\overline{\omega}) = \max\left\{-\rho \ln (\overline{\omega}/x(S,\overline{\omega})) + \rho x(S,\overline{\omega})^2/2, 0\right\}
\label{generalmuc}
\end{equation} %of \eqref{VFxc}
which is weakly increasing in $S$. Intuitively, when the no-absconding constraint holds with equality, the principal wishes to backload utility in order to relax the future no-absconding constraints, and so introduces a drift upwards in consumption. Note that by definition, we must have $S_{\textnormal{loc}}(\overline{\omega}) \leq 1/2$. Further, explicit manipulation given in Lemma \ref{upper} shows that this upper bound is attained at $\overline{\omega} = e^{1/2}$, and so the range of possible values of consumption volatility when the no-absconding holds as a strict inequality is the entire interval $[0, \sqrt{\rho}]$. To understand the relationship between the parameter $S$ and the volatility of consumption $\sqrt{\rho} x_{\textnormal{loc}}(S)$, we have the following, the proof of which is found in Appendix \ref{agencyAPP}. 

\begin{lemma}\label{Rfact}
The function $x_{\textnormal{loc}}$ satisfies $\lim_{S \rightarrow 0} x_{\textnormal{loc}}(S)/S = 1$ and for all $S \in [0,1/2]$ we have $S \leq x_{\textnormal{loc}}(S) \leq 2S$ and $x_{\textnormal{loc}}'(S) \geq 1$. 
\end{lemma}

Section \ref{DecenGenEq} shows how a class of stationary efficient allocations may be decentralized in a general equilibrium model using a particular set of taxes and transfers. Such a characterization is necessarily specific to the choice of Pareto weights attached to different generations and the assumed market structure. To isolate the role of agency frictions independently of general equilibrium effects, I will first analyze optimal wedges in partial equilibrium. If the return from continually investing in an asset over the interval $[t, t + \Delta]$ is $R = R_{t,t+\Delta}$, then intertemporal optimization implies $u'(c_t) = e^{-\rho \Delta} \mathbb{E}[Ru'(c_{t + \Delta}) | \mathcal{F}_t]$. The following measure the extent to which this optimization fails for an arbitrary return.

\begin{defn} \label{wedgedefn}
Given a consumption process $(c_t)_{t\geq0}$ and asset $A$ with return $(R^A_t)_{t\geq0}$ the associated wedge $\nu^A$ is defined by $u'(c_0) = e^{-\rho t}\mathbb{E}[ e^{-\nu^At} R^A_t u'(c_t)]$.
\end{defn} 

Denote by $\nu^K$ and $\nu^B$ the wedges associated with capital and the bond, respectively, and note that the associated returns are $R^K_t = e^{{\left(\Pi - \tau_I - \sigma^2/2\right)}t + \sigma B_t}$ and $R^B_t = e^{\rho t}$. These wedges represent the extent to which the presence of private information forces the technological returns on each asset to differ from the returns accruing to the entrepreneur. When the no-absconding constraint is satisfied as a strict inequality, the associated closed-form expression for consumption allows for a sharp characterization of these wedges. The proof of the following is contained in Appendix \ref{agencyAPP}. 

\begin{lemma} \label{LEMMAwedge}
If the no-absconding inequality is strict then the wedge on risky capital is $\nu^K = \Pi - \rho - \tau_I +\rho x_{\textnormal{loc}}(S)^2-\sqrt{\rho} \sigma x_{\textnormal{loc}}(S)$ and the wedge on the bond is $\nu^B = \rho x_{\textnormal{loc}}(S)^2$. Further, the wedge on the bond is always non-negative and the difference $\nu^B - \nu^K$ is both non-negative and increasing in $\Pi$. 
\end{lemma}

Lemma \ref{LEMMAwedge} shows that it is always efficient to distort the return on the entrepreneur's saving below the rate available to the principal, and that the magnitude of this distortion is an increasing function of $S$. However, the sign of the distortion on the risky asset is in general ambiguous, as simple examples show. For instance, if $S = 1/4$ and $\sqrt{\rho} = \sigma$, then $\nu^K/\rho = \phi/4 + (2 - \sqrt{3})^2 - (2 - \sqrt{3})$ ranges from approximately -0.2 to 0.05 as $\phi$ ranges from $0$ to $1$. Since the risky wedge vanishes with the excess return, it also is not monotonic in the marginal product of capital. However, although the wedge of the risky asset cannot be signed, it is always lower than the wedge on the bond and the difference is increasing in the marginal product of capital. This last observation will be important for the discussion of the optimal policy in the presence of heterogeneous entrepreneurs given in Section \ref{discuss}. 

Before turning to the environment with a continuum of entrepreneurs and an endogenous productivity of capital, it is useful to summarize the key insights that emerge from this partial equilibrium setting. First, whenever the principal's problem is finite-valued, the optimal contract takes a simple form and the growth of the entrepreneur's consumption exhibits constant mean and volatility. Second, the no-absconding constraint may or may hold not with equality in the optimal contract, and further, it either binds after every history, or never binds. Third, the risk borne by the entrepreneur is an increasing function of the ratio $S$ of the excess return on capital to the severity of the agency frictions. The primary task of the next section is to explain how these agency frictions determine the marginal product of capital when some agents (workers) work for others (entrepreneurs) and output is a constant-returns-to-scale function of capital and labor. Intuitively, when agency frictions are small relative to this excess return, the principal wishes to delegate more capital to the agent, which tends to increase the capital stock and therefore reduce the marginal product of capital. The net effect of these two forces on the key parameter $S$ is therefore ambiguous, and warrants further analysis.

\section{Stationary efficient allocations} \label{STATsection}

Section \ref{PAmodel} characterized the efficient contract between a risk-averse agent and a risk-neutral principal given an exogenous interest rate and productivity. This section uses the above to characterize a particular stationary efficient allocation in a production economy with a continuum of entrepreneurs and workers. Time is again continuous and extends indefinitely. At any moment there is a unit mass of agents who discount at rate $\rho_S$, die at rate $\rho_D$ and are endowed with $L$ units of labor. All agents have preferences represented by 
$$
U(c) := \mathbb{E}{\left[\rho\int_{0}^{\infty}e^{-\rho t}\ln c_tdt\right]}
$$ 
where $\rho := \rho_S + \rho_D$. Agents may either run a firm or work for someone else. However, only a fraction $1 - \psi \in [0,1]$ of each generation, termed entrepreneurs, are capable of running a firm, with the remaining fraction, termed workers, only able to work for someone else. I follow \cite{angeletos_uninsured_2007} and assume that these activities are not mutually exclusive and that entrepreneurs may perform both simultaneously. Whether an agent is an entrepreneur or a worker will be private information and referred to as their \textit{type} and indexed by $i\in \{E,W\}$. Entrepreneurs have access to a risky production technology that produces consumption using physical capital and labor. As in Section \ref{PAmodel}, an entrepreneur may abscond with a fraction $\iota$ of her capital and after doing so trade only a risk-free bond with return $\rho$, and they may also divert a flow of capital, with each unit diverted yielding $\phi \in [0,1]$ units of consumption. If capital and labor are assigned to an entrepreneur according to the processes $(k_t, l_t)_{t\geq0}$ and the entrepreneur adheres to the diversion strategy $(s_t)_{t\geq0}$ then output satisfies 
$$
dY_t = {\left(Ak_t^{\alpha}l_t^{1-\alpha} - \delta k_t - s_tk_t\right)}dt + \sigma k_t dB_t,
$$
where $B := (B_t)_{t\geq0}$ is a standard Brownian motion, $A>0$ and $\alpha \in (0,1)$ are exogenous constants and $\delta$ is the depreciation rate. Since the production function is constant-returns-to-scale in labor and capital, for a given wage the problem of the planner when facing an individual entrepreneur is isomorphic to the principal-agent problem of Section \ref{PAmodel}.

However, in contrast with Section \ref{PAmodel}, an allocation is now indexed by an initial distribution $\Phi$ over promised utility and types, and must specify the consumption, capital and labor delegated to an agent as a function of initial utility or date of birth, type, and the history of their output. Since agents supply labor inelastically, I will omit labor supply from the definition, and will also assume without loss that the planner never recommend an entrepreneur divert a positive amount of capital. In the following, $(k^{v,i}_t, l^{v,i}_t)$ and $(k^{T,i}_t, l^{T,i}_t)$ refer to the capital and labor assigned to a given type, where for agents not alive at the initial date, the superscript indicates their birth date. 

\begin{defn} \label{defnALLOCgen}
Given a distribution $\Phi$ over utility and types, an allocation consists of sequences $(c^{v,i}_t,k^{v,i}_t,l^{v,i}_t)_{t\geq 0}, (v,i) \in \textnormal{supp}(\Phi)$ for the initial generation and $(c^{T,i}_t, k^{T,i}_t,l^{T,i}_t)_{t\geq T \geq 0}$, $i = E,W$, for subsequent generations. An allocation satisfies promise-keeping if $U(c^{v,i}) = v$ for all $(v, i) \in \textnormal{supp}(\Phi)$, and is incentive-compatible if $U(c^{T,E}) \geq U(c^{T,W})$ for all $T \geq 0$ and the allocations to entrepreneurs satisfy the incentive compatibility conditions in Section \ref{PAmodel}.
\end{defn}

The planner need not worry about double deviations (in which the agent misreports type and then diverts output), since workers cannot pretend to be entrepreneurs and entrepreneurs who pretend to be workers are not entrusted with any capital and so thereafter have no private information. Denote by $C_t(A), K_t(A), Y_t(A)$ and $L_t(A)$ aggregate consumption, capital, output, and labor assigned at $t$ in allocation $A$.\footnote{Formal definitions are in Appendix \ref{ARESC}.} 

\begin{defn}
An allocation $A$ is resource feasible given capital $K$ if $K_0 = K$, $C_t(A) + \dot{K}_t(A) \leq Y_t(A)$ and $L_t(A) \leq L$ for all $t \geq 0$. An allocation is incentive feasible if it is both resource feasible and incentive-compatible, and the set of all such allocations will be denoted $\mathcal{A}^{IF}(\Phi,K)$.
\end{defn} 
% and the planner behaves like a principal for this marginal product of capital. 

I will assume that the planner cares only about workers and values the utility of a worker at any date the same regardless of their date of birth, which amounts to placing weight $e^{-\rho_ST}$ on the $T$th generation. The fact that the planner only values the utility of workers simplifies the analysis because it means that the planner just gives the entrepreneurs the least level of utility necessary to reveal their type. I also restrict attention to efficient allocations in which cross-sectional distributions are constant over time. The method by which this is achieved is similar to that followed in \cite{farhi_inequality_2007} and so details are relegated to Appendix \ref{reduct}. Essentially, one first relaxes the problem of the planner by considering the allocation they would choose if they could trade at the rate of discount. In this way, the law of motion of aggregate capital is replaced with constraint on the present value of resources, and the problem decomposes into many problems all identical in form to the principal-agent problem given in Definition \ref{Pprob}. If for some initial utility distribution and capital stock the planner would not wish to trade, then this present value constraint implies the resource constraint is satisfied every period.

Relative to the partial equilibrium setting in Section \ref{PAmodel}, the resource constraints and the presence of a continuum of agents affect the analysis in two ways. The production technology and the stock of labor jointly determine the marginal product of capital, while the fact that the planner places weight on future generations leads them to behave as if they faced a subsidy on capital. The problem of the planner facing a newborn is identical to the problem of the principal in Section \ref{PAmodel} in which $\tau_I = -\rho_D$ and the marginal product of capital is endogenous. This marginal product of capital $\Pi = \alpha A(K/L)^{\alpha-1} - \delta$, or equivalently, the ratio $S = (\Pi - \rho_S)/(\sqrt{\rho}\phi \sigma)$, must in turn vary until the goods resource constraint holds. One technical subtlety is that such an $S$ might fail to exist, since aggregate consumption and capital will diverge if the growth in utility exceeds the rate of death. In this paper I simply rule this possibility out by making the following assumption.\footnote{Proposition \ref{suffPROP} will show that this is satisfied for sufficiently small agency frictions.}

\begin{assump} \label{existSTAT}
The equation 
\begin{equation}
(1-\psi)\overline{C}(S) + \psi = {\left(S\sqrt{\rho}\phi \sigma/\alpha + \rho_S/\alpha + (1/\alpha-1)\delta \right)}(1-\psi)\overline{K}(S)
\label{statRCpsi}
\end{equation}
has a solution $\hat{S}$ satisfying $\mu_c(\hat{S},\overline{\omega}) < \rho_D$, where $\overline{C}(S)$ and $\overline{K}(S)$ are the aggregate amount of consumption and capital delegated to entrepreneurs per unit of initial utility in the stationary distribution, for a given level of $S$. 
\end{assump}

The explicit expressions for $\overline{C}(S)$ and $\overline{K}(S)$ are not important for what follows and so are relegated to Appendix \ref{STATapp}. The following characterizes the stationary efficient allocation for the above welfare notion and parameters. 

\begin{prop}\label{statINEQ}
A solution to equation \eqref{statRCpsi} is unique whenever Assumption \ref{existSTAT} is satisfied. In this case, an efficient stationary allocation exists in which the marginal product of capital is $\hat{\Pi} = \rho_S + \hat{S}\sqrt{\rho}\phi \sigma$ and the capital stock is $\hat{K} = (\alpha A/(\hat{\Pi} + \delta))^{\frac{1}{1-\alpha}}L$.
\end{prop}

The proof of Proposition \ref{statINEQ} essentially amounts to rearranging the goods resource constraint and is contained in Appendix \ref{STATapp}. Note that when the no-absconding constraint holds as a strict inequality, the resource constraint simplifies because the drift in the utility of entrepreneurs vanishes. Proposition \ref{suffPROP} below shows that this will be true when agency frictions are sufficiently small in the following sense. When varying the agency frictions, the parameters governing the diversion and absconding constraints will be assumed to be in fixed proportions to one another, so that $\iota = \phi \overline{\iota}$ for some $\overline{\iota} \in (0, 1]$ and all $\phi \in [0,1]$. Throughout this paper, all comparative statics with respect to $\phi$ will fix $\overline{\iota}$ in this manner, so that the full information environment corresponds to the limit $\phi \rightarrow 0$. 

\begin{prop} \label{suffPROP}
Assumption \ref{existSTAT} is satisfied for all sufficiently small agency frictions. The solution $\hat{S}$ is increasing in $\phi$ wherever it is well-defined and tends to zero as $\phi \rightarrow 0$, and so the no-absconding constraint holds as a strict inequality for all sufficiently small $\phi$.  
\end{prop}

\begin{proof}
Notice that the parameter $\overline{\omega} := \sqrt{\rho}\phi \sigma/(\rho \iota)$ remains fixed as we vary the agency frictions in the above manner. Rearranging the resource constraint \eqref{statRCpsi} then gives
$$
\alpha\sqrt{\rho}\sigma (\psi/\overline{C}(S) + 1 - \psi) = {\left((\rho_S + (1-\alpha)\delta)/\phi + S\sqrt{\rho}\sigma\right)}(1-\psi)x(S, \overline{\omega}).
$$
Both claims in the proposition are then immediate, because the right-hand side is decreasing in $\phi$ and diverges as $\phi \rightarrow 0$. 
\end{proof}

Proposition \ref{suffPROP} is noteworthy because the comparative statics in the general equilibrium environment are the exact opposite of those which obtain in the partial equilibrium setting of Section \ref{PAmodel}. When the marginal product of capital is fixed as in Section \ref{PAmodel}, the parameter $S$ governing the risk borne by the agent mechanically increases as $\phi$ falls. However, in the infinite-horizon setting with an aggregate production technology, a reduction in agency frictions increases the incentive to delegate capital to the agent, which tends to increase the capital stock and therefore reduces the marginal product of capital. Proposition \ref{suffPROP} shows that the latter force always overwhelms the partial equilibrium effect, so that the risk borne by agents is increasing in agency frictions. 

The simplicity of the characterization given in Proposition \ref{statINEQ} is due partly to the preferences being homothetic and partly to the adoption of a welfare criterion that weights the flow utility of an agent the same independently of her birth date. As emphasized in the partial equilibrium setting of Section \ref{PAmodel}, the homotheticity of preferences and the fact that technology exhibits constant-returns-to-scale implies that consumption and capital are linear in normalized utility. This permits aggregation over both entrepreneurs and workers and leads to the simple form of the resource constraint given above. Also note that the welfare notion I adopt differs from many other papers in the literature on dynamic contracting with private information, such as \cite{atkeson_efficient_1992} or \cite{phelan_incentives_1994}, who also consider component planner problems similar to the above generational planner's problems, but either assume zero discounting or place weight solely on the first generation. In the latter case this requires solving a component planning problem for a given interest rate that is then varied until resources are balanced. In contrast, with the welfare criterion of this paper, the only price for which stationarity may arise is the subjective discount rate of the agents, as all other prices induce a trend in consumption across generations. 

When the no-absconding constraint does not hold with equality, the stationary allocation is completely characterized by the initial levels of consumptions for both workers and entrepreneurs and the volatility of entrepreneurs' consumption. The partial equilibrium analysis of Section \ref{PAmodel} illustrates that in this case, the volatility of consumption growth must lie between 0 and $\sqrt{\rho}$. It is natural to then ask whether every value for volatility possible in the partial equilibrium setting can arise in the general equilibrium setting or whether some other stricter upper bound obtains. This will be relevant when discussing the range of taxes possible in the decentralization of Section \ref{DecenGenEq}. Substituting $S = 1/2$ into \eqref{statRCpsi} and assuming the no-absconding constraint does not hold with equality, we obtain
\begin{equation}
\alpha \sqrt{\rho}\phi \sigma(1 - \psi + \psi e^{-1/2}) = {\left(\rho_S + \sqrt{\rho}\phi \sigma/2 + (1-\alpha)\delta \right)}(1-\psi).
\label{statRCpsi4}
\end{equation}
Using Lemma \ref{upper}, if $\overline{\omega} = e^{1/2}$ and the general equilibrium parameters solve \eqref{statRCpsi4}, then the no-absconding constraint is satisfied and $\hat{x} = 1$. 

Before turning to the decentralization it is useful to summarize the main points of the above characterization. The efficient allocation is completely described by the following requirements: all newborns attain the same level of utility, workers have zero drift in consumption, entrepreneurs have drift $\mu_c(\hat{S},\overline{\omega})$ in consumption, the volatility of entrepreneurs' consumption is $\sqrt{\rho} x(\hat{S},\overline{\omega})$, and the capital stock is given by Proposition \ref{statINEQ}. The next section characterizes the taxes and intergenerational transfers that ensure these properties arise in a stationary competitive equilibrium with collateral constraints.  


\section{Decentralization} \label{DecenGenEq}

Section \ref{STATsection} characterized a stationary efficient allocation, with the distribution of resources conducted by a social planner executing the direct mechanism. To address the questions posed in the introduction I now consider how this allocation may be implemented with taxes and transfers when agents trade in decentralized markets. Such an analysis necessarily depends on the degree of risk-sharing in the private sector. At one extreme, if agents can write optimal long-term contracts with a competitive sector of intermediaries, the role of the government is relegated to distributing wealth across generations. This follows from arguments similar to those in \cite{atkeson_efficient_1992}, as the problem of an intermediary minimizing the cost of delivering utility is the dual of the problem of a planner maximizing utility subject to resource constraints. If intermediary profits are zero then the decentralized allocation corresponds to those of a planner who exactly exhausts all resources. Policy only calls for transfers to newborns and requires no taxes on agents during their lifetime. However, the assumption of such a sector is quite strong, as it may be too costly for business owners to find counterparties for such contracts or they may value privacy of their finances and direct control of their business. A government may therefore wish to provide additional social insurance using some combination of taxes and transfers. In accordance with optimal taxation analyses with unobservable labor productivity, I will first consider a market structure in which agents trade only a risk-free bond in zero net supply.

In this analysis it is essential that both the market structure and the choices of taxes and transfers by the government respect the informational asymmetries inherent in the environment. I will assume that the planner may transfer wealth lump-sum to newborn agents and levy constant linear taxes on savings and labor income, and that these choices may depend upon whether or not the agent chooses to become a worker or entrepreneur. Since types are private information, these transfers and taxes must be chosen so that the utility of entrepreneurs weakly exceeds that of workers. Entrepreneurs rent capital on behalf of their business at the risk-free rate and pay taxes on their firm's output net of payments to labor, interest payments and reported depreciation (profits). I emphasize that since entrepreneurs may divert a fraction of capital to consumption, this tax is levied on \textit{reported} profits, inclusive of any output diverted. Further, since the entrepreneurs may abscond with a fraction of the capital stock, capital will be subject to a collateral constraint, in which the amount rented by an entrepreneur cannot exceed a multiple of their wealth. The equilibrium notion will then impose the requirement that this constraint is the least restrictive value such that no individual ever wishes to abscond with their borrowed capital. Finally, all agents may contract with risk-neutral insurance companies to insure against longevity risks. Such longevity risks are common knowledge and unaffected by behavior, and so it seems natural to assume that they may be insured away. 

The main result is that for this market structure the stationary constrained-efficient allocation may be implemented using the above instruments. Further, the taxes on savings and profits play distinct roles and depend upon different economic forces. The tax on profits provides risk-sharing, while the tax on savings is chosen to ensure that the degree of consumption smoothing coincides with the efficient level. The tax on profits depends only on the misreporting friction, while the tax on savings will depend upon general equilibrium effects governing the response of the real interest rate to incomplete markets. 

For each type $i \in \{E,W\}$, I will write $\eta_i$ for the fraction of aggregate capital received at birth, and $\tau_{si}$ and $\tau_{Li}$ for the taxes on savings and labor income, respectively. All agents may borrow up to the natural borrowing limit, and so the absence of labor income risk implies that the relevant state variable is total wealth, the sum of bond holdings and the present discounted after-tax value of labor income. As a fraction of the aggregate capital stock, the initial total wealth per unit of aggregate capital of an agent of type $i$ is denoted $\kappa_i = \eta_i + h_i$, where $h_i$ is human wealth and given by
$$
h_i = \frac{(1-\tau_{Li})wL/K}{(1-\tau_{si})(r + \rho_D)}.
$$
Labor is inelastically supplied and agents can borrow up to their natural debt limit, and so taxes on labor income are equivalent to lump-sum transfers. For this reason, the decentralization that follows will be stated in terms of $\kappa_i$ rather than $\eta_i$. For brevity I will also write $r_i = (1-\tau_{si})(r+\rho_D)$ for the after-tax return on savings of type $i$. 

\begin{defn}\label{defnPROB}
Given the wage $w$, marginal product of capital $\Pi$, risk-free rate $r$ and collateral constraint $\hat{\omega}$, the problem of an entrepreneur with assets $a$ and human wealth $h_E$ facing taxes $\tau_{sE}$ and $\tau_{\Pi}$ is
\begin{align*}
V_E(a) = & \max_{\substack{(c_t,k_t,s_t)_{t\geq0}}}  \mathbb{E}{\left[\rho\int_{0}^{\infty}e^{-\rho t} \ln (c_t + \phi s_tk_t)dt\right]}
\\ da_t & = {\left[r_Ea_t - c_t + (1-\tau_{LE})wL\right]}dt + (1-\tau_{\Pi})k_tdR(s_t)_t
\\ k_t & \leq \hat{\omega} (a_t + h_E)
\\ 0 & \leq a_t + h_E 
\end{align*}
where $dR(s_t)_t = (\Pi - r - s_t)dt + \sigma dB_t$. The problem of a worker with assets $a$ and human wealth $h_W$ facing taxes $\tau_{sW}$ is
\begin{align*}
V_W(a) = & \max_{\substack{(c_t)_{t\geq0}}} \mathbb{E}{\left[\rho\int_{0}^{\infty}e^{-\rho t} \ln c_t dt\right]}
\\ da_t & = {\left[r_Wa_t - c_t + (1-\tau_{LW})wL\right]}dt
\\ 0 & \leq a_t + h_W. 
\end{align*}
\end{defn}

Although it is of a standard form, there are several aspects of the entrepreneur's problem in Definition \ref{defnPROB} that are worth emphasizing. First, since the net return on investment in the business $dR_t$ may be negative, the above formulation embodies the assumption that the agent receives a tax offset if her firm sustains losses. Second, the return on the entrepreneur's savings is $r_E$ but the borrowing cost of the firm is simply $r$, which captures the fact that $\tau_{\Pi}$ is a tax on firm profits rather than a tax on firm revenue, because interest, wages and depreciation are subtracted prior to the imposition of the tax. In particular, it is assumed that the owner pays herself a wage equal to the marginal product of her labor. Third, the tax on profits provides risk-sharing between the government and the entrepreneur. To see this, note that the flow tax revenue from an entrepreneur's firm is $\tau_{\Pi}(\Pi - r)k_tdt + \tau_{\Pi}k_t\sigma dZ_t$, and so with a positive profits tax, the government receives a fraction $\tau_{\Pi}$ of the shock $k_t\sigma dZ_t$ to capital, with only the remainder $1 - \tau_{\Pi}$ passing through to the entrepreneur. Stealing affects only the mean return on investment, and so the optimal strategy of the agent is independent of wealth and investment and solves 
$$
\max_{s \in [0,\overline{s}]}\ \phi s + (1-\tau_{\Pi})(\Pi - r - s).
$$
The agent will choose $s=0$ if and only if $\tau_{\Pi} \leq 1 - \phi$. Under this assumption, the problem of the entrepreneur admits a simple characterization. 

\begin{lemma} \label{agentprobLEMMA}
The entrepreneur will choose not to divert capital if and only if $\tau_{\Pi} \leq 1-\phi$, in which case their value function is $V_E(a) = \ln \rho + \ln{\left(a + h_E\right)} + \rho^{-1} {\left(\mu_a - \sigma_a^2/2\right)}$, where 
\begin{equation}
\begin{aligned}
\mu_a & = r_E - \rho + (1-\tau_{\Pi}) (\Pi - r) \hat{k} 
\ \ \ \ \ \ & \ \ \ \ \ \ \sigma_a & = (1-\tau_{\Pi}) \sigma \hat{k} 
\label{phiMU}
\end{aligned}
\end{equation} 
and the constant $\hat{k}$ is given by
%\begin{equation}
$$
\hat{k} = \min{\left\{ \frac{\Pi - r}{\sigma^2(1-\tau_{\Pi})}, \hat{\omega}\right\}}.
$$ %\label{khat}\end{equation}
The policy function for consumption is $c(a) = \rho (a + h_E)$ and the policy function for capital is $k(a) = \hat{k}(a + h_E)$. 
\end{lemma}

\begin{proof}
See Appendix \ref{agentprobproof}. 
\end{proof}

In the entrepreneur's problem in Definition \ref{defnPROB}, the parameter $\hat{\omega}$ in the collateral constraint is an arbitrary constant. Since the presence of the collateral constraint is motivated by the no-absconding constraint, in what follows $\hat{\omega}$ will be taken to be the largest value such that the agent will never wish to abscond. Using Lemma \ref{agentprobLEMMA}, this is equivalent to
\begin{equation}
\hat{\omega} = \iota^{-1}e^{{\left(\mu_a - \sigma_a^2/2\right)}/\rho}. 
\label{hatomega}
\end{equation}
The collateral constraint depends on tax policy because this affects the utility from not absconding. The following is the notion of equilibrium I adopt in this paper. 

\begin{defn}
Given taxes $\tau_{si}, \tau_{Li}$ and $\tau_{\Pi}$ and transfers $\eta_i$ for $i \in \{E,W\}$, a stationary competitive equilibrium consists of a capital stock $K$, wage $w$, risk-free rate $r$, and collateral parameter $\hat{\omega}$, such that agents solve the problems in Definition \ref{defnPROB}, markets for labor, capital, and goods clear, the government budget constraint is satisfied, the utility of entrepreneurs weakly exceeds that of workers, and $\hat{\omega}$ satisfies \eqref{hatomega}. 
\end{defn}

The following is the main result of this paper. It shows that whenever the stationary efficient allocation characterized in Section \ref{STATsection} exists, it may be implemented with linear taxes on savings and profits, together with transfers to newborn agents. 

\begin{prop} \label{zerodriftEQ}
Under Assumption \ref{existSTAT}, the stationary efficient allocation can be implemented as a competitive equilibrium with linear taxes on savings and profits. Writing $\hat{x} = x(\hat{S}, \overline{\omega})$, the interest rate is $r = \hat{\Pi} - \sqrt{\rho} \sigma \hat{x}$, the tax on profits is $\tau_{\Pi} = 1 - \phi$, and the taxes on savings are % of entrepreneurs and workers are
\begin{align*}
1-\tau_{sE} & = \frac{\rho(1 - \hat{x}^2) + \mu_c(\hat{x})}{r + \rho_D} &
1-\tau_{sW} & = \frac{\rho}{r + \rho_D}.
\end{align*}
The endowed wealth of entrepreneurs and workers as a fraction of the capital stock is
\begin{align*}
\kappa_E & = \frac{\phi \sigma(\rho_D - \mu_c(\hat{x}))}{\sqrt{\rho} \hat{x}(1-\psi)\rho_D} &
\kappa_W & = \kappa_E\max\{e^{-\hat{x}^2/2}, \hat{x}/\overline{\omega}\}
\end{align*}
and the constant in the collateral constraint is $\hat{\omega} = \iota^{-1}e^{\mu_c(\hat{x})/\rho - \hat{x}^2/2}$. 
\end{prop}

%Finally, the interest rate is always below $\rho_S$, and so the tax on workers' savings is everywhere negative. 


The proof of Proposition \ref{zerodriftEQ} is contained in Appendix \ref{decentDT}. Although the expressions for transfers and the collateral constraints are complicated, the logic behind the proof is quite simple. From Lemma \ref{agentprobLEMMA}, the risk borne by the agent when $\tau_{\Pi} = 1 - \phi$ and the collateral constraint does not bind is $(\Pi - r)/(\phi\sigma)$, which will equal the efficient value $\sqrt{\rho}\hat{x}$ if and only if the interest rate is given by the above expression. The optimal transfers to entrepreneurs and workers are then chosen so that the bond market clears at this value. The after-tax return on workers' savings is equal to the discount rate because they possess no private information. The entrepreneur earns a return of $\rho \hat{x}^2$ on their business and so the mean growth in their wealth and consumption is $r_E - \rho + \rho\hat{x}^2$, which equals the efficient level $\mu(\hat{x})$ for the above choice of taxes. 


Also note that government policy is still necessary when agency frictions vanish, because the direct weight placed on future generations leads to intergenerational transfers. There is some indeterminacy here because lump-sum transfers have the same effect as taxes on labor when the latter is inelastically supplied. If taxes on labor vanish, then the goods resource constraint gives $\kappa_E \approx (\rho_S/\alpha + (1/\alpha-1)\delta)/\rho$ and the wealth inherited by agents is $\kappa_EK - wL/\rho \approx \rho_SK/\rho$. Proposition \ref{statINEQ} shows that $\hat{S} \approx \hat{x} \approx 0$, and so the savings taxes are approximately zero and the risk-free rate is approximately equal to the subjective discount rate. The tax on profits remains $1 - \phi$ but the revenue collected is negligible because profits approach zero. Tax revenue therefore vanishes and the government owns a fraction $\rho_D/\rho$ of the capital stock, the interest on which is transferred to newborns. 

Notice that when the no-absconding constraint holds as a strict inequality, the expressions in Proposition \ref{zerodriftEQ} imply $r_E - r_W = - \rho \hat{x}^2 < 0$. Consequently, although all agents face constant type-dependent linear taxes, in some qualified sense the model of this paper implies progressivity of savings taxes, as entrepreneurs (who are typically wealthier) will earn a lower return on their savings than workers.\footnote{Note that for some parameters the pre-tax return on saving $r + \rho_D$ might be negative, in which case we have $\tau_{sE} < \tau_{sW}$ and an increase in the ``taxes'' increases the return on savings.} 

Finally, Lemma \ref{LEMMAwedge} shows that in the partial equilibrium setting, the principal always wishes to distort the agent's return on savings below the risk-free rate available to the principal. However, Proposition \ref{zerodriftEQ} shows that the risk-free rate is below the subjective rate of time preference, and so it is not clear whether the tax on savings of the entrepreneurs is positive or negative. Indeed, as the following shows, both possibilities can occur. The proof may be found in Appendix \ref{decentDT}. 

\begin{lemma} \label{negS}
For any $\sigma>0$ the savings tax on entrepreneurs is negative for all sufficiently small $\phi$. However, for $\phi = 1$ the savings tax is positive for $\sigma < \sqrt{\rho}$ when the no-absconding constraint does not hold with equality.
\end{lemma}


\section{Discussion and robustness}\label{discuss}

The proof of Proposition \ref{zerodriftEQ} amounts to ensuring that the capital stock, initial consumption and the mean and volatility of consumption growth in the competitive equilibrium coincide with their counterparts in the efficient allocation. I now discuss its robustness to various extensions and also provide some intuition for the role played by each instrument.

\textbf{Mechanism design and the primal approach.} First, it is worth emphasizing that the mechanism design approach adopted above simplifies the characterization of the optimal taxes. This may appear counterintuitive, since modeling incentive constraints necessitates the analysis of a dynamic agency problem apparently unrelated to the incomplete markets model of Section \ref{DecenGenEq}. However, proceeding in this way eliminates the need to understand how competitive equilibria vary with taxes. The government can never do better than the efficient allocation when choosing taxes (because the equilibrium allocations are incentive compatible), and so the task of Proposition \ref{zerodriftEQ} is simply to show that the efficient allocation can be implemented with taxes and transfers, which only involves solving a system of simultaneous equations. If taxes and transfers were the objects of choice in the planner's problem, then the analysis would be more complicated, since the interest rate, capital stock and the constant appearing in the collateral constraint are only defined in terms of the solutions to market clearing equations. A change in any instrument will have non-obvious effects on all of these objects, but this is irrelevant to the proof of Proposition \ref{zerodriftEQ}.

This reasoning is reminiscent of the primal approach employed in the literature on optimal linear taxation in representative agent economies.\footnote{See, e.g., \cite{chari_optimal_1999}.} Recall that here one rearranges the first-order conditions of the consumer's problem to eliminate prices from the budget constraint and thereby obtain what is termed an ``implementability condition''. One can then reverse this procedure and show that any allocation that is resource feasible and satisfies the implementability condition can be supported as a competitive equilibrium. In this way, there is no need to understand exactly how competitive equilibria vary with taxes, and the planner's problem becomes a standard programming problem. The analogy with the approach of this paper is far from exact, but in both cases one uses the optimality conditions that obtain in competitive equilibria (here these conditions just imply constant mean and volatility of consumption growth) and then chooses among allocations directly.

\textbf{The role of the profits tax.} The fact that a change in one instrument will, in general, affect all equilibrium quantities makes it difficult to assign a unique role to each instrument. However, we can gain some intuition by varying a particular instrument while others are held fixed at arbitrary values. The entrepreneur will choose not to divert output to consumption if and only if $\tau_{\Pi} \leq 1 - \phi$, and so the prescription $\tau_{\Pi} = 1 - \phi$ is intuitively obvious. Further, as the discussion following Definition \ref{defnPROB} shows, this tax leads to risk-sharing between the entrepreneur and the government. The choice of the profits tax therefore ensures that risk-sharing is at the highest degree possible given the informational asymmetry inherent in the environment. However, the manner in which this tax affects the risk borne by the entrepreneurs is subtle and depends upon general equilibrium forces. 

Lemma \ref{agentprobLEMMA} shows that investment is increasing in the profits tax whenever the collateral constraint does not bind. This follows from the constant returns in production and the symmetric treatment of profits and losses, and illustrates an effect first highlighted in \cite{domar_proportional_1944}. By levying the profits tax, the government essentially becomes a partner in the business, receiving a constant share of both profits and losses. However, in partial equilibrium, the tax does not reduce the risk borne by the entrepreneur or affect their after-tax return on wealth (unless the collateral constraint binds), because the entrepreneur simply increases their leverage. \cite{boadway_optimal_2021} make a similar observation in a two-period model in which agents face exogenous and different expected returns on risky assets, and conclude that such a tax serves no redistributive role. However, in the analysis of the current paper, in which the marginal product of capital is endogenously determined by resource constraints and agency frictions, there is an additional general equilibrium effect that requires this view to be qualified. To see this, suppose that the savings taxes are chosen so that the inverse Euler equation holds and that the collateral constraints do not bind, but that all other instruments are arbitrary. Then the bond market clearing condition becomes $K = (1 - \psi) \kappa_E K \hat{k}$, and so in equilibrium, the volatility of consumption growth is
\begin{align*}
(1 - \tau_{\Pi}) \sigma \hat{k} & = \frac{(1-\tau_{\Pi}) \sigma}{(1-\psi) \kappa_E}
\end{align*}
which is in fact decreasing in the profits tax. In other words, when the tax on profits is high (but incentive compatible), leverage is high, which drives down the marginal product of capital and the excess return on the risky asset in the long-run. In general equilibrium, the tax on profits may therefore still reduce the difference in returns experienced by entrepreneurs and workers even in the presence of the releveraging effect of \cite{domar_proportional_1944}.

Also note that although the profits tax depends solely on the ability of the agent to divert resources, it is not imposed to discourage such diversion. Indeed, in the absence of such a tax, the entrepreneur would have no desire to misreport income, since she is the residual claimant of output and so any misreporting would only lose her money, due to the presence of the deadweight loss. If the profits tax were below the efficient level, the problem would not be that entrepreneurs now engage in socially wasteful activities, but that instead risk-sharing and investment would be too low. 

\textbf{Alternative market structures.} In the above I assumed that agents could only trade a risk-free bond in zero net supply. There are at least two directions in which this assumption could be relaxed. First, one could allow for the existence of limited private risk-sharing in the form of equity markets, so that the government's fiscal policy is not the only source of risk-sharing. Indeed, although \cite{smith_capitalists_2019} show that business ownership is highly concentrated, the assumption in the above that every firm is owned by a single individual is extreme and unrealistic. When entrepreneurs can issue equity, the allocations without taxes now coincide with the environment of \cite{di_tella_uncertainty_2017}, who shows that when entrepreneurs can write optimal short-term contracts with a competitive sector of risk-neutral intermediaries, the sole effect of the contracting is to reduce their exposure to their firm's output by the factor of $\phi$. This is the lowest level consistent with incentive compatibility, and so the private contracting plays the same risk-sharing role as the profits tax in Section \ref{DecenGenEq}, which is now redundant. If the intermediary sector is competitive, it makes no profits and so the mean excess return on capital is $\hat{\Pi} - r$ and the problem of the entrepreneur is identical to that in Definition \ref{defnPROB} with $\tau_{\Pi} = 0$ and $\sigma$ replaced with $\phi \sigma$. Reasoning identical to Proposition \ref{zerodriftEQ} implies that the real interest rate in the optimal allocation is now $r = \hat{\Pi} - \sqrt{\rho} \phi \sigma \hat{x}$, while the expressions for the savings taxes in terms of this interest rate remain unchanged. 

As a second alternative to the benchmark market structure, we could allow the risk-free asset to be a government bond with exogenous return. For instance, suppose that the government allowed agents to borrow and lend at the pre-tax value $r = \rho_S$. Now there would be no need to tax or subsidize workers' savings to ensure that their consumption remained constant, while the tax on entrepreneurs' savings would have to rise in order to keep their after-tax return on the bond unchanged. However, if the firm borrowed at $r = \rho_S$, then the risk borne by the owner would be $(\hat{\Pi} - \rho_S)/\sigma$, which is strictly below $\sqrt{\rho} \hat{x}$, the efficient level of consumption risk, and so the entrepreneur would not have sufficient incentives to invest. To decentralize the efficient allocation the government must now subsidize the borrowing of the firm so that their after-tax borrowing costs are $\hat{\Pi} - \sqrt{\rho}\sigma\hat{x}$. 

\textbf{Heterogeneous entrepreneurs.} The above remarks also help us to understand how the efficient allocation and taxes must change when entrepreneurs differ in their productivity and the riskiness of their projects. Incorporating unobservable heterogeneity would substantially increase the complexity of the analysis and is beyond the scope of this paper, but if such heterogeneity is observable the preceding analysis carries over quite easily. In this case, the marginal product of capital will in general differ among entrepreneurs, and the analogue of the resource constraint in Proposition \ref{statINEQ} becomes more complicated.\footnote{Appendix \ref{hetero_ent} provides details of the characterization. For the purpose of the this discussion the point is simply that projects with different marginal products of capital may be operative in the efficient allocation.} However, the arguments employed in Proposition \ref{zerodriftEQ} remain applicable provided that the government allows taxes on savings and firm borrowing to depend upon productivity and risk. An entrepreneur with marginal product of capital $\Pi$ will bear the efficient level of risk provided that the after-tax cost of their firm's borrowing is $r_B(\Pi) = \rho_S - (\nu^B - \nu^K)$, where $\nu^K$ and $\nu^B$ are the wedges defined in Section \ref{PAmodel} and I wrote $S(\Pi) \equiv (\Pi - \rho_S)/(\sqrt{\rho}\phi\sigma)$. Lemma \ref{LEMMAwedge} then implies that $r_B(\Pi)$ is decreasing in $\Pi$, and so heterogeneity in productivity provides a force for regressivity, as the planner wishes to increase the profits of more productive entrepreneurs by subsidizing borrowing. However, if the collateral constraints do not bind, all entrepreneurs earn the same expected return on their wealth because consumption is a martingale, and the higher profits of the more productive entrepreneurs are exactly offset by a lower after-tax return on the bond. Further, the optimal profits tax remains common to all entrepreneurs at $\tau_{\Pi} = 1 - \phi$, unless there is some reason why more productive entrepreneurs find it easier to misreport and divert output.

The case of heterogeneous entrepreneurs also allows us to understand how the above results change when we relax the assumption that all production is subject to uninsurable idiosyncratic risk. In quantitative macroeconomic models with entrepreneurs, such as \cite{quadrini_entrepreneurship_2000} and \cite{cagetti_entrepreneurship_2006}, one typically assumes that only some of the output is produced by privately-run firms, with the remaining output produced by a corporate sector operating a less productive technology. This is simply a special case of the situation with heterogeneous entrepreneurs in which the agency frictions affecting one type of entrepreneur vanish. One way to understand this situation is to suppose that we began with the allocation in Section \ref{DecenGenEq} with $\phi \sigma = 0$, in which both the marginal product of capital and the interest rate are equal to the subjective rate of discount $\rho_S$, and then add a population of entrepreneurs subject to agency frictions who operate a more productive technology. If the marginal product of capital of the entrepreneurs were $\overline{\Pi} > \rho_S$, then the borrowing of these entrepreneurs would have to be subsidized by the amount $\rho_S - r_B(\overline{\Pi}) > 0$ and the tax on savings would be $x(S(\overline{\Pi}), \overline{\omega})^2 - \mu(x(S(\overline{\Pi}), \overline{\omega}))/\rho$. 

\textbf{Relationship to literature.} In light of the simplicity of Proposition \ref{zerodriftEQ}, it is natural to wonder how it relates to other results in the literature. I now compare the findings with \cite{albanesi_optimal_2006} and \cite{shourideh_optimal_2013}, who both adopt the mechanism design approach to study optimal taxation in the presence of multiple sources of capital income. 

First, in \cite{albanesi_optimal_2006}, entrepreneurs incur an additively separable effort cost which affects the expected return on investment. The amount of capital delegated to the entrepreneur does not enter the incentive constraints, unlike in the current paper in which a high level of capital makes diversion more attractive. Further, \cite{albanesi_optimal_2006} considers a two-period environment with fixed marginal product of capital, which precludes an analysis of how taxes ought to depend on history or the extent to which taxes and agency frictions affect the excess return on capital in the long-run. 

Second, although both \cite{shourideh_optimal_2013} and the current paper allow entrepreneurs to divert capital to consumption, I focus on a decentralization with incomplete markets. \cite{shourideh_optimal_2013} finds that the wedge on the risk-free asset is increasing in productivity but that the wedge on the risky asset exhibits no such monotonicity, two observations that echo Lemma \ref{LEMMAwedge}. However, in the decentralization in Section \ref{DecenGenEq}, neither of these wedges corresponds to the profits tax, which serves to partially correct for missing equity markets and depends only on the agency friction. Further, the above discussion shows that the efficient borrowing cost depends on the difference between the risky and risk-free wedges. Although in the efficient allocation more productive entrepreneurs may or may not face larger wedges on the risky asset, their firms unambiguously face lower borrowing costs and unchanged profits taxes. 

% is levied on output minus the wage bill, depreciation and interest payments, and the latter is endogenously determined by bond market clearing and therefore indirectly affected by government transfers.
%Albanesi notes that entrepreneurial capital is subsidized relative to the other forms of capital income. 
%direct comparison is difficult because I have considered a continuous-time environment, there are some obvious differences worth emphasizing. For the agency problem, the key difference seems to be that in \cite{shourideh_optimal_2013} consumption can be diverted before investment, so that diversion affects not just the mean but also the variance of output. This ensures that there is no analogue of the simple characterization of incentive compatible profits taxes in Lemma \ref{agentprobLEMMA}. However

The mechanism design approach here also allows us to complement recent results in the Ramsey literature. \cite{panousi_optimal_2012} study optimal capital taxation in an environment similar to that in this paper but levy a common tax on all capital income. The presence of idiosyncratic risk depresses the risk-free rate, and \cite{panousi_optimal_2012} show that the sign of the optimal long-run tax on capital income is ambiguous and depends upon the strength of this precautionary motive relative to the insurance (or redistributive) objective of the government. For low levels of risk, redistributive concerns are minimal and the tax on capital is negative, while the reverse is true for high levels of risk. In the current paper, profits and savings may be taxed at different rates, and we have qualitatively different comparative statics results. There is now no ambiguity in the sign of the profits tax, which is always non-negative, while Lemma \ref{negS} shows that when agency frictions are high, the savings tax is always non-negative for low levels of risk.


\section{Numerical examples}\label{numerical}

The goal of this paper has been to study the optimal taxation of profits and savings in a model in which the economic mechanisms are as transparent as possible. In this section I compute some numerical examples to illustrate the theory.\footnote{I focus on the benchmark decentralization given in Section \ref{DecenGenEq}. Appendix \ref{figuresAPP} computes analogous quantities for the decentralization with private equity markets discussed in Section \ref{discuss}.} I will fix
\begin{equation} %corresponds to an annual discount factor of roughly $0.96$,
(\alpha, \sigma, \rho_S, \rho_D, \delta) = (0.33, 0.2, 0.04, 0.02, 0.05)
\label{standardPARAMS}
\end{equation} %$ is consistent with an expected working life of forty-five years and $
and will explore how taxes and equilibrium quantities vary with the severity of the agency frictions considered in this paper.\footnote{The figures in this section differ from the example in Figure 2 of the 2019 version both due to different parameter values but also to a coding error that was subsequently corrected.} The parameters $\rho_D, \rho_S$ and $\delta$ are all standard, while $\alpha$ and $\sigma$ are conservatively chosen to be in the lower end of the range of values typically adopted in the literature. 

Figure \ref{tax} plots the taxes on savings for different levels of $(\phi, \psi)$ and $\overline{\iota}=1$, which generates the tightest possible collateral constraint when agency frictions are high.\footnote{For context, \cite{cagetti_entrepreneurship_2006} show that approximately 11.5\% of households in the SCF are active business owners, which corresponds to the highest value $\psi = 0.885$.} As expected from Lemma \ref{negS}, both workers and entrepreneurs face negative taxes on savings for small values of the agency frictions. 

%and Proposition \ref{zerodriftEQ}
\begin{figure}[H]
\centering
\caption{Savings taxes with tight collateral constraints ($\overline{\iota}=1.0$)}
\includegraphics[width=0.49\linewidth, height=0.35\linewidth]{taxes_entrepreneurs1.0}
\includegraphics[width=0.49\linewidth, height=0.35\linewidth]{taxes_workers1.0}
\label{tax} 
\end{figure} 

Figure \ref{r_muc} depicts the expected consumption growth of entrepreneurs and the real interest rate, and shows that the ``kinks'' in Figure \ref{tax} occur at points where the planner begins to backload consumption in order to relax the future no-absconding constraint. Figure \ref{r_muc} also illustrates that although the interest rate is always below the subjective discount rate, it is not monotonic in agency frictions. Recall that Proposition \ref{zerodriftEQ} shows that the interest rate is the difference between the marginal product of capital and the product of the exogenous volatility with the volatility of consumption growth. Both of these forces are increasing in agency frictions, and Figure \ref{r_muc} shows that the net effect on the interest rate is in general ambiguous. Because the tax on workers is simply chosen so that the consumption of workers is constant, it inherits this lack of monotonicity. 

\begin{figure}[H]
\centering
\caption{Expected consumption growth and the interest rate}
\includegraphics[width=0.49\linewidth, height=0.35\linewidth]{return_wealth1.0}
\includegraphics[width=0.49\linewidth, height=0.35\linewidth]{r1.0}
\label{r_muc} 
\end{figure} 

Figure \ref{tax2} complements Figure \ref{tax} by depicting the taxes on savings under parameters for which the collateral constraints are as relaxed as possible, which from Lemma \ref{upper} occurs when $\overline{\iota} \approx 0.5$. The taxes are identical to those in Figure \ref{tax} for low agency frictions (where collateral constraints do not bind), but are much larger for high agency frictions. 

\begin{figure}[H]
\centering
\caption{Savings taxes with relaxed collateral constraints ($\overline{\iota}=0.5$)}
\includegraphics[width=0.49\linewidth, height=0.35\linewidth]{taxes_entrepreneurs}
\includegraphics[width=0.49\linewidth, height=0.35\linewidth]{taxes_workers}
\label{tax2} 
\end{figure} 

These magnitudes may appear surprising in light of the literature on optimal linear capital taxation, in which there is typically a zero tax on capital in the long-run.\footnote{Note that these large taxes are not due to extreme choices of $\alpha$ and $\sigma$, because these are at the lower end of their typical ranges and higher values will only increase the effect of agency frictions.} There are many reasons for the difference, but perhaps the most obvious one is that here the capital income of entrepreneurs is the sum of their savings and the profits of their business, and so a low return on the risk-free bond does not imply a low total return on capital. For instance, when $(\psi, \phi) = (0.885, 1)$, so that the taxes in Figure \ref{tax2} are at their highest value, the entrepreneurs consume 6\% of their wealth per period and earn a return of over 5\% from their business. In order for the inverse Euler equation to hold, the after-tax return on the bond must fall below 1\%.

This illustrates a simple but powerful effect in this model: the (symmetric) tax on profits does not reduce after-tax business income, and so the optimal tax on interest income will grow large as agency frictions rise even though the expected after-tax return on capital remains unchanged at the complete markets value. When the excess return on capital is high, the planner simply wants most of the entrepreneur's income to assume the form of profits rather than interest. However, it is worth emphasizing that in the above allocation entrepreneurs must receive lump-sum transfers to compensate them for the risk they bear, for otherwise they would have no incentive to reveal their type and start a firm.  

The high sensitivity of taxes to parameters in this environment is why I have simply computed a range of examples instead of emphasizing the values that obtain under a benchmark set of parameters. \cite{smith_capitalists_2019} show that there exists substantial heterogeneity in returns on private businesses, and there is much we do not know regarding the determinants of this income. In particular, although business income appears to reflect owner-specific characteristics, the extent to which these are endogenous to tax policy is not yet clear. In this paper I have therefore emphasized the distinct roles played by each instrument, roles that are less sensitive to the precise form of the production technology. I interpret the above examples as providing suggestive evidence that quantitative work ought to allow for the possibility of levying different taxes on various forms of capital income, both because empirical work indicates that such heterogeneity is quantitatively significant and because its presence provides a simple additional force for the taxation of savings. 


\section{Conclusion} \label{conc}

The United States tax code currently levies different taxes on interest income and business profits. This paper has provided a tractable model in which the desirability of this differential taxation emerges when business owners may divert resources to private consumption and can abscond with a fraction of assets. I show that whenever a stationary efficient allocation exists, it may be implemented with lump-sum transfers together with constant, occupation-specific, linear taxes on profits and savings. 

The principal findings regarding these taxes and transfers were as follows. First, the profits tax depends solely on the degree of frictions in financial markets and is symmetric in the sense that it allows for full offset of losses. It serves only to share risk and is redundant when entrepreneurs can issue equity. Second, when the collateral constraint does not bind, entrepreneurs face lower after-tax returns on savings than workers, and so the model generates progressive savings taxes in a qualified sense. Third, the rate at which businesses borrow must fall below the complete markets value in order to provide entrepreneurs with the incentive to invest at the efficient level. When agents trade a bond in zero net supply, this is achieved via a reduction in the real interest rate, which necessitates a subsidy on workers' savings. Finally, when the model is extended to allow for heterogeneous entrepreneurs, the optimal profits tax remains common to all agents, the savings tax rises with productivity, and the investment of more productive entrepreneurs must be subsidized. 

\bibliography{OpDiffCap}

\appendix
\small

\section{Agency problem}\label{agencyAPP}
\subsection{Incentive compatibility} \label{agency_IC}

The characterization of incentive compatibility adopted in Section \ref{PAmodel} follows from the arguments employed in the online appendix to \cite{di_tella_optimal_2019}. However, because the situation in the current paper is not quite a special case of \cite{di_tella_optimal_2019}, I will spell out some additional details.

Given an elasticity of intertemporal substitution $\psi$ and risk aversion parameter $\gamma$, \cite{di_tella_optimal_2019} considers an intermediary (analogous to what I have termed an entrepreneur) subject to a cash diversion problem as in the current paper but without the no-absconding constraint, with the utility process $(U_t)_{t\geq0}$ from consumption $(c_t)_{t\geq0}$ satisfying
\begin{equation}
U_t := \mathbb{E}_t{\left[ \int_t^{\infty} f(c_s, U_s) ds \right]}
\label{EZint}
\end{equation}
where the Epstein-Zin aggregator is defined by 
\begin{equation}
f(c, U) := \frac{\rho}{1-1/\psi} {\left(\frac{c^{1-1/\psi}}{[(1-\gamma)U]^{\frac{\gamma-1/\psi}{1-\gamma}}} - (1-\gamma) U \right)}.
\label{EZf}
\end{equation}
CRRA utility corresponds to $\psi = 1/\gamma$ and logarithmic utility arises as $\gamma, \psi \rightarrow 1$. For the CRRA case we have $f(c, U) := \rho{\left(c^{1-\gamma}/(1-\gamma) - U \right)}$ and for the logarithmic case we have $f(c, U) := \rho{\left(\ln c - U \right)}$. 

\begin{lemma} \label{ICchar} 
For any allocation $(k,c)$, the law of motion for promised utility admits the representation $dW_t = \rho{\left(W_t - \ln c_t\right)}dt + \tilde{\sigma}_{W,t} dB_t$ for some process $\tilde{\sigma}_{W,t}$. The allocation is incentive compatible if and only if 
\begin{equation}
0 \in \textnormal{arg} \max_{s\geq 0} \rho \ln(c_t + \phi k_t s) - s \tilde{\sigma}_{W,t} /\sigma - \rho \ln c_t
\label{localIC}
\end{equation}
almost surely for all $t\geq0$, and so when characterizing efficient allocations there is no loss in assuming that utility evolves according to
\begin{equation}
dW_t = \rho{\left(W_t - \ln c_t\right)}dt + \rho \phi \sigma (k_t/c_t) dB_t.
\label{IC_rep}
\end{equation}
\end{lemma}

\begin{proof}
This follows from Lemma 1 and the proof of Lemma 2 in \cite{di_tella_optimal_2019}. Note that although the literal statement of Lemma 2 in \cite{di_tella_optimal_2019} does not apply to logarithmic utility, the proof does in fact extend to this case, and indeed to all CRRA utility functions, because the only point at which \cite{di_tella_optimal_2019} uses properties of the aggregator in \eqref{EZf} is for the existence of a constant $\kappa$ such that $f(c,y) - f(c,x) \leq \kappa(y - x)$ for any $c$ and all $y \geq x$. For general Epstein-Zin preferences, this requires some restrictions on parameters, such as those imposed in \cite{di_tella_optimal_2019}, but for the case of CRRA utility, we have $f(c, y) - f(c, x) = \rho(-y+x) \leq 0$ for $y \geq x$, and so this holds automatically.
\end{proof}

Using Ito's Lemma we have the following law of motion for utility in consumption units. 

\begin{corl} \label{corl_IC}
When characterizing efficient allocations, there is no loss in assuming that utility in consumption units evolves according to the diffusion process
$$
du_t = \rho{\left( -\ln(c_t/u_t) + \frac{1}{2}(\sqrt{\rho} \phi  \sigma k_t/c_t)^2\right)}u_tdt + (\rho \phi \sigma k_t/c_t)u_tdB_t. 
$$
\end{corl}

\subsection{Characterization of value function}

To prepare for the following proofs I will introduce some additional notation. I will define $\overline{x}$ and $\overline{\overline{x}}$ to be the solutions to $\overline{x}e^{\overline{x}^2/2} = \overline{\omega}$ and $\overline{\overline{x}}e^{\overline{\overline{x}}^2/2-1} = \overline{\omega}$, respectively. Note that $\overline{x}$ is the maximum $x$ for which the no-absconding constraint holds, under the assumption of the inverse Euler equation holding, and $\overline{\overline{x}}$ is the maximum $x$ for which consumption growth is smaller than the rate of discount when the no-absconding constraint holds with equality. 

Recall that Proposition \ref{existence} asserts that the problem of the principal is finite-valued for all sufficiently small $S$. The technical conditions defining ``sufficiently small'' are the following, where $\overline{v} \equiv \overline{v}(S, \overline{\omega})$ is defined in \eqref{VFxc}. 

\begin{assump}\label{ASSsuff}
$S\overline{\overline{x}}<1$. 
\end{assump}

\begin{assump}\label{ASSsuff2}
$S\overline{\omega} + \rho(1 + S^{-2})\overline{v}(S, \overline{\omega}) < 0$.
\end{assump}

\begin{lemma} \label{vneg}
If Assumption \ref{ASSsuff} holds, then $\overline{v}(S, \overline{\omega}) \leq 0$. 
\end{lemma}

\begin{proof}
I will assume the existence of a pair $(\overline{c}, x)$ in the constraint set such that the maximand in the definition of $\overline{v}$ is positive and derive a contradiction. If such a pair existed, then the maximand would also be positive at $(\overline{\omega}/x, x)$, which would imply an $x \geq0$ such that $1 + \ln (\overline{\omega}/x) - x^2/2 > 0$ and $Sx > 1$, which violates Assumption \ref{ASSsuff}. 
\end{proof}

\begin{proof}[Proof of Proposition \ref{existence}]
For the purpose of this proof, write $\overline{S}_1(\overline{\omega})$ for any return such that Assumption \ref{ASSsuff} is satisfied if $S \leq \overline{S}_1(\overline{\omega})$ and $\overline{S}_2(\overline{\omega})$ for any return such that Assumption \ref{ASSsuff2} is satisfied if $S \leq \overline{S}_2(\overline{\omega})$. The following will show that the return in the statement of the proposition can be taken to be $\overline{S}(\overline{\omega}) = \min\{\overline{S}_1(\overline{\omega}), \overline{S}_2(\overline{\omega})\}$. 

Using the characterization of incentive compatibility given in Lemma \ref{ICchar}, the problem of the principal can be viewed as a control problem in which the state is $W$ and evolves according to the law of motion $dW_t = \rho(W_t - \ln c_t)dt + \rho \phi \sigma (k_t/c_t)dB_t = - \rho \ln \overline{c}_tdt + \sqrt{\rho} x_tdB_t$ when the control variables are $\overline{c}_t$ and $x_t$. By the general theory of continuous-time, finite-horizon dynamic programming outlined in \cite{kushner_numerical_2001}, the value function $V^T$ solves %the terminal value problem
\begin{align*}
0 & = \sup_{\substack{\overline{c},x\geq 0 \\ x\overline{c} \leq \overline{\omega}}} (Sx - 1)\overline{c}e^W - \rho\ln \overline{c} \frac{\partial V^T}{\partial W} + \frac{\rho x^2}{2}\frac{\partial^2V^T}{\partial W^2} + \frac{\partial V^T}{\partial t} - \rho V
\end{align*} %$x = \sqrt{\rho} \phi \sigma kc^{-1}$ and
where $V^T(W,T) = -\rho^{-1}e^W$ and I changed variables to $c = \overline{c}e^W$. Using the same homogeneity argument as in Section \ref{PAmodel}, we know that the value function is of the form $V^T(W,t) = \overline{V}_{T-t}e^{W}$ for some function $s \mapsto \overline{V}_s$ satisfying $\overline{V}_0 = -\rho^{-1}$. Substituting and dividing by $e^W$, the above partial differential equation reduces to the ordinary differential equation $\dot{\overline{V}}_s = J(\overline{V}_s)$ for $s \in [0,T]$ with the initial condition $\overline{V}_0 = -\rho^{-1}$, where $\dot{\overline{V}}_s$ is the time derivative of $\overline{V}_s$, and
\begin{equation}
J(V) = \sup_{\substack{\overline{c}, x\geq 0 \\ x\overline{c} \leq \overline{\omega}}} (Sx - 1)\overline{c} + \rho( -\ln \overline{c} + x^2/2 - 1)V.
\label{Jdef}
\end{equation} 
Note that $J$ is convex as it is the pointwise maximum of linear functions. I will show that the maintained assumptions ensure that $J$ has a root at $V = \overline{v}$, or 
$$
0 = \sup_{\substack{\overline{c},x\geq 0 \\ x\overline{c} \leq \overline{\omega}}} (Sx - 1)\overline{c} + \rho(-\ln\overline{c} + x^2/2 - 1)\overline{v} =: \sup_{\substack{\overline{c},x\geq 0 \\ x\overline{c} \leq \overline{\omega}}} H(\overline{c}, x, \overline{v})
$$
where the second equality defines the function $H$. If $(\overline{c}^*, x^*)$ are the maximizers in \eqref{VFxc}, then $H(\overline{c}^*, x^*, \overline{v}) = 0$. It remains to eliminate the possibility that $H(\overline{c}, x, \overline{v}) > 0$ for some $(\overline{c}, x)$ satisfying $\overline{c},x\geq 0$ and $x\overline{c} \leq \overline{\omega}$. The existence of such a pair $(\overline{c}, x)$ that also satisfied $- \ln \overline{c} + x^2/2 - 1 < 0$ would violate the definition of $\overline{v}$, and so it will suffice to rule out the existence of $(\overline{c}, x)$ satisfying $H(\overline{c}, x, \overline{v}) > 0$, $\overline{c}x \leq \overline{\omega}$ and $- \ln \overline{c} + x^2/2 - 1 > 0$. Because $\overline{v} < 0$ by Lemma \ref{vneg}, the inequality $H(\overline{c}, x, \overline{v}) > 0$ requires $Sx > 1$, and so would imply $(Sx - 1)\overline{\omega}/x + \rho(-\ln(\overline{\omega}/x) + x^2/2 - 1)\overline{v} > 0$. It will then suffice to show that this last inequality is impossible under the assumption $S < \overline{S}(\overline{\omega})$, and so I want to show that
\begin{equation}
(Sx - 1)\overline{\omega}/x + \rho(-\ln(\overline{\omega}/x) + x^2/2 - 1)\overline{v} \leq 0
\label{notGUAR}
\end{equation}
for all $x \geq \overline{\overline{x}}$. The inequality \eqref{notGUAR} holds automatically on $x \in [\overline{\overline{x}}, 1/S]$ from Assumption \ref{ASSsuff}, and so it will therefore suffice for the derivative of the left-hand side to be negative for all $x \geq 1/S$. This is equivalent to $\overline{\omega}/x + \rho(1 + x^2)\overline{v} \leq 0$ for $x \geq 1/S$. Since $\overline{v}<0$, this will be true for $x \geq 1/S$ if and only if it is true for $x = 1/S$, which is exactly Assumption \ref{ASSsuff2}. 

The above argument shows that the function $J$ has a root $\overline{v}$, which will then be the coefficient of the stationary solution to the above Hamilton-Jacobi-Bellman equation. It remains to show that the value function indeed converges to this root as the horizon becomes infinite. To this end, note that for any scalars $C_1$ and $C_2$, the terminal value problem
\begin{align*}
f'(s) & = -\kappa (f(s) - C_1) 
\\ f(0) & = C_2
\end{align*}
has solution $f(s) = (C_2 - C_1)e^{-\kappa s} + C_1$, which converges to $C_1$ as $s \rightarrow \infty$ for any $\kappa>0$. It therefore suffices to bound $V$ between the solutions $\check{V}$ and $\hat{V}$ to the initial value problems $\dot{\check{V}}_s = J'(\overline{v})(\check{V}_s - \overline{v})$ and $\dot{\hat{V}}_s = J'(-\rho^{-1})(\hat{V}_s - \overline{v})$ with initial conditions $\check{V}_0 = \hat{V}_0 = -\rho^{-1}$, which both converge to $\overline{v}$ as $s \rightarrow \infty$. This in turn implies that 
\begin{equation}
\check{V}_s \leq V_s \leq \hat{V}_s
\label{sandwich}
\end{equation}
for all $s \geq 0$. To see \eqref{sandwich}, note that if, to the contrary, we had $\check{V}_{s_1} > V_{s_1}$ at some $s_1 > 0$, then, by the fundamental theorem of calculus, at some $s_2 > s_1$ we would have $\check{V}_{s_2} = V_{s_2}$ and $\dot{\check{V}}_{s_2} > \dot{V}_{s_2}$, or
$$
J'(\overline{v})(V_{s_2} - \overline{v}) > J(V_{s_2}) = J(V_{s_2}) - J(\overline{v})
$$
which violates the convexity of $J$. The other inequality in \eqref{sandwich} may be established similarly, and so by taking limits we have $\lim_{s \rightarrow \infty} V_s = \overline{v}$. 
\end{proof}

\begin{proof}[Proof of Proposition \ref{suffLEM}]
From Proposition \ref{existence}, wherever the principal's problem is well-defined, the optimal choices of $\overline{c}$ and $x$ solve
\begin{equation}
\overline{v} = \max_{\substack{\overline{c},x\geq 0, x\overline{c} \leq \overline{\omega} \\ -\ln \overline{c} + x^2/2<1}} \ \frac{(Sx - 1)\overline{c}}{\rho(1 + \ln \overline{c} - x^2/2)}.
\label{VFxc2}
\end{equation}
Since the above maximand is negative in the constraint set wherever it is well-defined, for each fixed $x$ the choice of consumption $\overline{c}$ will always solve
\begin{align*}
\min_{\substack{\overline{c} \geq 0, x\overline{c} \leq \overline{\omega} \\ -\ln \overline{c} + x^2/2<1}} \frac{\overline{c}}{1 + \ln \overline{c} - x^2/2}.
\end{align*}
Changing variables to $C = \ln \overline{c}$ and taking logarithms (which leaves extrema unaffected), we need to minimize $C - \ln(1 + C - x^2/2)$ over the set of real $C$ satisfying $xe^C \leq \overline{\omega}$ and $-C + x^2/2<1$. Since this latter minimand is convex and diverges as $C \rightarrow x^2/2-1$ from above, the optimal choice either occurs at the solution to the first-order condition or the boundary point $C = \ln(\overline{\omega}/x)$. In term of the original variables, the principal's problem becomes
\begin{equation}
\overline{v} = \max_{x \in [0, \overline{\overline{x}}]} \ \frac{(Sx - 1)\min\{e^{x^2/2},\overline{\omega}/x\}}{\rho(1 + \min\{0,\ln (\overline{\omega}/x) - x^2/2\})}
\label{VFxc3}
\end{equation}
where I remind the reader that $\overline{\overline{x}}$ is defined to be the solution to $\overline{\overline{x}}e^{\overline{\overline{x}}^2/2-1} = \overline{\omega}$. The no-absconding constraint will not bind if the optimal $x$ in \eqref{VFxc3} lies in $[0, \overline{x}]$, where $\overline{x}$ solves $\overline{x}e^{\overline{x}^2/2} = \overline{\omega}$. By Topkis' theorem, the optimal $x$ will be increasing in $S$ because the cross-partial derivative with respect to $S$ and $x$ of the maximand in \eqref{VFxc3} is positive, and the claims in the statement all follow. 
\end{proof}

\begin{lemma} \label{upper}
The function $S_{\textnormal{loc}}$ satisfies $S_{\textnormal{loc}}(e^{1/2}) = 1/2$.
\end{lemma}

\begin{proof} %[Proof of Lemma \ref{upper}] 
I want to show that $1 = x_{\textnormal{loc}}(1/2) = x(1/2, e^{1/2})$. Because $\overline{c} = e^{1/2}$ when $x=1$, we require
$$
-e^{1/2}/2 \geq \max_{x\in [1,\overline{\overline{x}}]} \frac{(x/2-1)e^{1/2}/x}{3/2 - \ln x - x^2/2}
$$
which is equivalent to $-(3/2 - \ln x - x^2/2) \geq (x-2)/x$, or $5/2 \leq \ln x + x^2/2 + 2/x$ for $x\in [1,\overline{\overline{x}}]$. This last inequality reduces to $5/2 \leq 5/2$ at $x=1$, and $(\ln x + x^2/2 + 2/x)' = 1/x + x - 2/x^2 = (x + x^3 - 2)/x^2$, which is non-negative on $x\geq1$.
\end{proof}

\begin{proof}[Proof of Lemma \ref{Rfact}] %x_{\textnormal{loc}}(S) = (1 - \sqrt{1 - 4R^2})/(2R)
Using $x_{\textnormal{loc}}(S) := 1/[2S] - \sqrt{1/[4S^2] - 1}$, we have 
\begin{align*} 
x_{\textnormal{loc}}'(S) & = -\frac{1}{2S^2} + \frac{1}{4S^3} \frac{1}{\sqrt{1/[4S^2] - 1}}
 = \frac{1}{2S^2} {\left( \frac{1}{\sqrt{1 - 4S^2}} - 1 \right)}.
\end{align*}
The inequality $x_{\textnormal{loc}}'(S) \geq 1$ is then equivalent to $1 \geq (1 + 2S^2)\sqrt{1 - 4S^2}$, and by squaring both sides, this in turn is equivalent to 
\begin{align*} 
1 & \geq (1 + 4S^2 + 4S^4)(1 - 4S^2) %= 1 + 4R^2 + 4R^4 - 4R^2(1 + 4R^2 + 4R^4)
 = 1 - 12S^4 - 16S^6 
\end{align*}
which is always true. The inequality $S \leq x_{\textnormal{loc}}(S)$ is equivalent to $2S^2 \leq 1 - \sqrt{1 - 4S^2}$, or $1 - 4S^2 \leq 1 - 4S^2 + 4S^4$, while $x_{\textnormal{loc}}(S) \leq 2S$ is equivalent to $1 - 4S^2 \leq \sqrt{1 - 4S^2}$, and both of these are true when $S \in [1/2]$. Finally, using l'Hopital's rule we have 
\begin{align*}
\lim_{S \rightarrow 0} x_{\textnormal{loc}}(S)/S & = \lim_{S \rightarrow 0} \frac{1 - \sqrt{1 - 4S^2}}{2S^2}
 = \lim_{S \rightarrow 0} \frac{4S/\sqrt{1 - 4S^2}}{4S} = 1
\end{align*}
as claimed. 
\end{proof}

\begin{proof}[Proof of Lemma \ref{LEMMAwedge}]
Since consumption satisfies $\ln(c_t/c_0) = -\rho x_{\textnormal{loc}}(S)^2t/2 + \sqrt{\rho} x_{\textnormal{loc}}(S) dB_t$, substituting $R^K$ into Definition \ref{wedgedefn} and taking logarithms gives
\begin{align*}
\nu^K & = \Pi - \rho - \tau_I - \sigma^2/2 + \rho x_{\textnormal{loc}}(S)^2/2 + \frac{1}{t}\ln \mathbb{E}{\left[ e^{(\sigma - \sqrt{\rho} x_{\textnormal{loc}}(S))B_t}\right]}.
\end{align*}
Using $\mathbb{E}[e^{zB_t}] = e^{z^2t/2}$ gives $\nu^K=\Pi-\rho-\tau_I+\rho x_{\textnormal{loc}}(S)^2 - \sqrt{\rho} \sigma x_{\textnormal{loc}}(S)$. Similarly, substitution of $R^B$ gives $\nu^B = \rho x_{\textnormal{loc}}(S)^2$. Since $\phi \leq 1$, the difference between the wedge on the bond and the risky wedge satisfies $\nu^B - \nu^K \geq \sqrt{\rho} \phi \sigma {\left(x_{\textnormal{loc}}(S) - S\right)}$ which is non-negative by Lemma \ref{Rfact}. Finally, the derivative of $\nu^B - \nu^K$ with respect to $\Pi$ is 
\begin{align*}
\sqrt{\rho} \sigma x_{\textnormal{loc}}'(S) \times \partial S/\partial \Pi - 1 \geq x_{\textnormal{loc}}'(S) - 1
\end{align*}
which is again non-negative by Lemma \ref{Rfact}.
\end{proof}


\section{Stationary efficient allocations} \label{STATdecent}
\subsection{Aggregate resource constraints}\label{ARESC}

%No $1 - \psi$ factor in initial generation, as this is captured by $\Phi$. i.e. we already have $1-\psi = \int_{\mathbb{R}}\Phi(dv, E) $

Aggregate quantities at any date are comprised of contributions from the initial generation and subsequent generations. Superscripts indicate date-of-birth (if not alive at the initial date) or promised utility (if alive at initial date). Aggregate quantities associated with the initial generation are distinguished by an underline, and aggregate quantities associated with the generation born at date $T$ are distinguished by a $T$ superscript. To understand the following, note that, e.g. $k^{T,E}_t$ is the capital assigned to an entrepreneur at $t$ born at date $T$, and so the total capital assigned to all such entrepreneurs (conditional on being alive) is $(1-\psi)\mathbb{E}[k^{T,E}_t]$. Average consumption and capital assigned at $t\geq0$ are 
\begin{align*} 
\underline{C}_t : & = \int_{\mathbb{R} \times \{E,W\}}\mathbb{E}[c^{v,i}_t]\Phi(dv, i), 
\ \ \ C^T_t := (1-\psi)\mathbb{E}[c^{T,E}_t] + \psi \mathbb{E}[c^{T,W}_t]
\\ C_t : & = e^{- \rho_D t}\underline{C}_t  + \rho_D\int_{0}^{t}e^{-\rho_D (t-T)} C^T_tdT
\\ \underline{K}_t : & = \int_{\mathbb{R}}\mathbb{E}[k^{v,E}_t]\Phi(dv, E), 
\ \ \ K^T_t := (1-\psi)\mathbb{E}[k^{T,E}_t]
\\ K_t : & = e^{- \rho_D t}\underline{K}_t + \rho_D\int_{0}^te^{-\rho_D (t-T)} K^T_tdT
\end{align*}
while output is 
\begin{align*}
\underline{Y}_t : & = \int_{\mathbb{R}}\mathbb{E}[F(k^{v,E}_t, l^{v,E}_t) - \delta k^{v,E}_t]\Phi(dv, E) 
\\ Y^T_t : & = (1 - \psi) \mathbb{E}[F(k^{T,E}_t, l^{T,E}_t) - \delta k^{T,E}_t]
\\ Y_t : & = e^{-\rho_D t}\underline{Y}_t + \rho_D\int_0^te^{-\rho_D (t-T)} Y^T_tdT 
\end{align*}
where $F(K,L) := AK^{\alpha}L^{1-\alpha}$. Aggregate labor assigned to entrepreneurs is
\begin{align*}
\underline{L}_t : & = \int_{\mathbb{R}}\mathbb{E}[l^{v,E}_t]\Phi(dv, E), \ \ \ L^T_t := (1-\psi) \mathbb{E}[l^{T,E}_t]
\\ L_t : & = e^{- \rho_D t}\underline{L}_t  +  \rho_D\int_0^te^{-\rho_D (t-T)} L^T_tdT.
\end{align*}
For future reference, note that for any function $H(T,t)$, using $e^{-\rho (t-T)}e^{-\rho_S T} = e^{-\rho_S t} e^{-\rho_D (t-T)}$ and interchanging the order of integration gives
\begin{equation}
\int_0^{\infty}\int_0^t  e^{-\rho_S t} e^{-\rho_D (t-T)}H(T,t)dtdT = \int_0^{\infty}\int_T^{\infty} e^{-\rho (t-T)}e^{-\rho_S T}H(T,t)dtdT.
\label{intint}
\end{equation}
It follows that the present discounted value of consumption when the interest rate is $\rho_S$ is given by 
\begin{align*}
\int_0^{\infty} e^{-\rho_St} C_t dt & = \int_0^{\infty} e^{-\rho_St} {\left(e^{- \rho_D t}\underline{C}_t + \rho_D\int_0^te^{-\rho_D(t-T)} C^T_tdT\right)} dt
\\ & = \int_0^{\infty} e^{-\rho t} \underline{C}_t dt + \rho_D \int_0^{\infty} \int_0^t e^{-\rho_St} e^{-\rho_D (t-T)} C^T_tdT dt
\\ & = \int_0^{\infty} e^{-\rho t} \underline{C}_t dt + \rho_D \int_0^{\infty} e^{-\rho_S T} \int_T^{\infty} e^{-\rho (t-T)} C^T_tdtdT
\end{align*}
and similarly for output and labor. This will be relevant for the decomposition by generation given below. The average flow utility experienced by the initial and $T$th generations at $t \geq 0$ is
\begin{align*}
\underline{U}^i_t & = \int_{\mathbb{R}} \mathbb{E} [\rho \ln(c^{v,i}_t)]\Phi(dv, i) &
U^{T,i}_t & = \mathbb{E} [\rho \ln(c^{T,i}_t)]
\end{align*}
for $i \in \{E,W\}$. Recall that in the main text I assume that the planner cares only about workers and values the utility of a worker at any date the same regardless of their date of birth. Since entrepreneurs are no more capable of providing labor than workers, they earn no information rent, and the sole restriction necessary for the truthful revelation of type is that they receive high lifetime utility than workers. Since the planner weights all flow utility the same regardless of an agent's date of birth, their objective function is
\begin{equation}
U^P = \int_0^{\infty}{\left(e^{-\rho t}\underline{U}^W_t + \rho_D \int_0^te^{-\rho_S T}\psi e^{-\rho  (t-T)} U^{T,W}_tdT\right)}dt.
\label{simpOBJ0}
\end{equation}
Using \eqref{intint}, the objective of the planner \eqref{simpOBJ0} becomes 
\begin{equation} %integrand in following is $e^{-\rho_ST} \psi W^{T,W}$ 
U^P = \int_{\mathbb{R}} v\Phi(dv,W) + \rho_D\int_0^{\infty}e^{-\rho_ST} \psi \int_T^{\infty}e^{-\rho (t-T)} \mathbb{E} [\rho \ln(c^{T,i}_t)]dt dT.
\label{simpOBJ}
\end{equation}

%W^{T,i} : & = \int_T^{\infty}e^{-\rho (t-T)} \mathbb{E} [\rho \ln(c^{T,i}_t)]dt for $i \in \{E,W\}$.

\begin{defn}\label{planprob}
Given $(\Phi, K)$, the planner's problem is $V(\Phi,K) = \max_{A \in \mathcal{A}^{IF}(\Phi,K)} U^P(A)$.
\end{defn}

In this paper I restrict attention to stationary solutions to the planner's problem. I therefore search for the distribution $\Phi$ and capital stock $K$ such that the distributions of utility, consumption and capital implied by the solution to Definition \ref{planprob} are constant over time. 

\subsection{Reduction to principal-agent problem}\label{reduct}

I will characterize stationary efficient allocations using the ideas outlined in \cite{farhi_inequality_2007} and consider, in succession, \emph{relaxed} and \emph{generational} planner's problems. The relaxed problem differs from the planner's problem by allowing intertemporal trade at rate $\rho_S$.

\begin{defn}\label{RELAXplanprob} 
Given $(\Phi,K)$, the relaxed planner's problem is 
\begin{align*}
V^R(\Phi,K) = \max_{A \in \mathcal{A}^{IC}(\Phi)} & U^P(A)
\\ \int_0^{\infty} e^{-\rho_S t}[C_t(A) + \dot{K}_t(A)]dt & \leq \int_0^{\infty} e^{-\rho_S t}Y_t(A)dt
\\ \int_0^{\infty} e^{-\rho_S t}L_t(A)dt & \leq \int_0^{\infty} e^{-\rho_S t}Ldt 
\\ K_0 & = K.
\end{align*}
\end{defn} %(since depreciation is positive and output is decreasing-returns-to-scale)
If an allocation solves the relaxed planner's problem and the distributions of utility and capital are constant over time, then it also solves the planner's problem beginning at that distribution and capital. To characterize stationary solutions to the original planner's problem, it therefore suffices to consider problems of the form in Definition \ref{RELAXplanprob} and find $\Phi$ and $K$ such that stationarity arises. The relaxed planner's problem therefore has only two resource constraints instead of two for each instant in time. Further, since $K_t$ remains bounded, integrating by parts implies that
$$
\int_0^{\infty}e^{-\rho_S t}\dot{K}_t(A)dt = -K_0(A) + \rho_S \int_{0}^{\infty}e^{-\rho_S t}K_t(A)dt. 
$$ %I will refer to the problem of dealing with a single generation as a \emph{generational} planner's problem. For a given $T$, this amounts to choosing $(c^{T,i}_t, k^{T,i}_t,l^{T,i}_t)_{t\geq T \geq 0}$ for $i = E,W$. 
Given a distribution $\Phi$ over utility and types, when the planner discounts at rate $\rho_S$ the relaxed problem in Definition \ref{RELAXplanprob} is then 
\begin{align*}
V^R(\Phi) = \max_{A \in \mathcal{A}^{IC}(\Phi)} \int_0^{\infty} & {\left(e^{-\rho t}\underline{U}^W_t + \rho_D \int_0^te^{-\rho_S T}\psi e^{-\rho  (t-T)} U^{T,W}_tdT\right)}dt.
\\ \int_0^{\infty} e^{-\rho_S t}[C_t(A) + \rho_SK_t(A) - Y_t(A)&]dt \leq K_0(A)
\\ \int_0^{\infty} e^{-\rho_S t}[L_t(A)-L&]dt \leq 0. 
\end{align*} %We now simply form the Lagrangian. The component planning problems are concave whenever the problems are well-defined. 
Denote by $\lambda_R$ and $\lambda_R\lambda_L$ the multipliers on the two resource constraints. The Lagrangian for the relaxed problem is
\begin{align*}
\mathcal{L} & = \int_{\mathbb{R}} v\Phi(dv,W) + \rho_D\int_0^{\infty}e^{-\rho_ST} \psi \int_T^{\infty}e^{-\rho (t-T)} \mathbb{E} [\rho \ln(c^{T,W}_t)]dt dT
\\ & - \lambda_R \int_0^{\infty} e^{-\rho_S t}[C_t + \rho_SK_t - Y_t + \lambda_LL_t]dt + \lambda_RK_0 + \lambda_R\lambda_LL.
\end{align*}
Using \eqref{intint} once again, the terms that do not depend on the initial generation are 
\begin{align*}
 & \int_0^{\infty}e^{-\rho_ST} \psi \int_T^{\infty}e^{-\rho (t-T)} \mathbb{E} [\rho \ln(c^{T,W}_t)]dt dT
\\ & - \lambda_R \int_0^{\infty} e^{-\rho_ST}\int_T^{\infty} e^{-\rho (t-T)} [C^T_t + \rho_SK^T_t - Y^T_t + \lambda_LL^T_t]dtdT
\\ & = \int_0^{\infty} e^{-\rho_ST}\int_T^{\infty}e^{-\rho (t-T)} {\left(\psi\mathbb{E} [\rho\ln(c^{T,W}_t)] - \lambda_R[C^T_t + \rho_SK^T_t - Y^T_t - \lambda_LL^T_t]\right)}dt dT .
\end{align*}
The term in parentheses may be written as 
\begin{align*}
 & \psi\mathbb{E} {\left[\rho\ln(c^{T,W}_t) - \lambda_R c^{T,W}_t\right]} + (1-\psi) \lambda_R \mathbb{E}{\left[{\left(A(l^{T,E}_t/k^{T,E}_t)^{1-\alpha} - \lambda_L(l^{T,E}_t/k^{T,E}_t) - \delta - \rho_S\right)}k^{T,E}_t - c^{T,E}_t\right]}.
\end{align*}
Now define 
\begin{equation}
\Pi(\lambda_L) := \max_{l\geq 0} A l^{1-\alpha} - \lambda_Ll - \delta = \alpha A^{1/\alpha} [(1-\alpha)/\lambda_L]^{1/\alpha-1} - \delta
\label{PilambdaL}
\end{equation}
and 
\begin{equation}
S(\lambda_L) = \frac{\Pi(\lambda_L) - \rho_S}{\sqrt{\rho}\phi \sigma}. % = \frac{\alpha A(K/L)^{\alpha-1} - \delta - \rho_S}{\sqrt{\rho}\phi \sigma},
\label{GER}
\end{equation}
It follows that the problem of the planner facing the $T$th generation just becomes 
\begin{equation}
\max_{\substack{W^E, W^W \in \mathbb{R} \\ W^E \geq W^W}} \psi W^W - \psi \lambda_R e^{W^W} + \lambda_R(1-\psi)\overline{v}(S(\lambda_L), \overline{\omega})e^{W^E}. 
\label{Tgen}
\end{equation}
Since $\overline{v}<0$ wherever it is well-defined, it is immediate that in the relaxed planner problem we have $W^E = W^W$ and so the problem reduces to
$$
\max_{W \in \mathbb{R}} \psi (W - \lambda_R e^{W}) + \lambda_R(1-\psi)\overline{v}(S(\lambda_L), \overline{\omega})e^{W}. 
$$

\subsection{Stationary resource constraint} \label{STATapp}

Given \eqref{GER}, the average consumption and capital delegated of entrepreneurs in the stationary distribution per unit of initial utility are
\begin{equation}
\begin{aligned} %note that $\overline{k}(S, \overline{\omega}) = \overline{c}x/(\sqrt{\rho}\phi sigma)$
\overline{C}(S) & = \frac{\rho_D\overline{c}(S, \overline{\omega})}{\rho_D - \mu_c(S,\overline{\omega})} 
\\ \overline{K}(S) & = \frac{\rho_D\overline{c}(S, \overline{\omega})x(S, \overline{\omega})}{(\rho_D - \mu_c(S,\overline{\omega}))\sqrt{\rho}\phi \sigma} 
\end{aligned}
\label{bigCK}
\end{equation}
where $\overline{c}$ and $x$ are the policy functions in the principal-agent problem from Section \ref{PAmodel}. For the change of variables adopted in Section \ref{PAmodel}, capital may be written as $\overline{k}(S,\overline{\omega}) := \overline{c}(S,\overline{\omega})x(S,\overline{\omega})/(\sqrt{\rho}\phi \sigma)$. The optimal labor-capital ratio from \eqref{PilambdaL} is $l(\lambda_L) = [(1-\alpha)A/\lambda_L]^{1/\alpha}$, and output per unit of capital is
\begin{equation}
\begin{aligned}
A l^{1-\alpha} - \delta & = A[(1-\alpha)A/\lambda_L]^{1/\alpha-1} - \delta
%\\ & = A^{1/\alpha} [(1-\alpha)/\lambda_L]^{1/\alpha-1} - \delta
\\ & = \Pi(\lambda_L)/\alpha + (1/\alpha - 1)\delta.
\label{outputPI}
\end{aligned}
\end{equation}

\iffalse
\begin{equation}
\Pi(\lambda_L) := \max_{l\geq 0} A l^{1-\alpha} - \lambda_Ll - \delta = \alpha A^{1/\alpha} [(1-\alpha)/\lambda_L]^{1/\alpha-1} - \delta
\label{PilambdaL}
\end{equation}
The optimal labor-capital ratio is $l(\lambda_L) = [(1-\alpha)A]^{1/\alpha}\lambda_L^{-1/\alpha}$ and the marginal product of capital is $\Pi(\lambda_L) = \alpha(1-\alpha)^{1/\alpha-1}A^{1/\alpha}\lambda_L^{1-1/\alpha} - \delta$. These last two expression imply 
$(\Pi(\lambda_L) + \delta)/\alpha = A[(1-\alpha)A]^{1/\alpha-1}\lambda_L^{1-1/\alpha} = Al(\lambda_L)^{1-\alpha}$.
\fi

\begin{proof}[Proof of Proposition \ref{statINEQ}] 
Using \eqref{outputPI}, the flow production net of depreciation from the firm of an entrepreneur with utility $u$ is then 
\begin{align*} %[(1-\alpha)A_j]^{1/\alpha}\lambda_L^{-1/\alpha}. (S(\lambda_L),\overline{\omega})
(Al(\lambda_L)^{1-\alpha} - \delta)\overline{k}(S(\lambda_L), \overline{\omega})u & = (\Pi(\lambda_L)/\alpha + (1/\alpha - 1)\delta)\overline{k}(S(\lambda_L), \overline{\omega})u.
\end{align*}
Aggregate consumption in the stationary distribution when the initial utility level is $u_0$ is then given by $((1-\psi)\overline{C}(S(\lambda_L)) + \psi)u_0$ and aggregate output is 
$(\Pi(\lambda_L)/\alpha + (1/\alpha - 1)\delta)(1-\psi)\overline{K}(S(\lambda_L))u_0$ and so canceling $u_0$ and substituting $S(\lambda_L) = (\Pi(\lambda_L) - \rho_S)/(\sqrt{\rho} \phi \sigma)$ gives the claimed expression. Given this $\lambda_L$, the capital stock $K$ is determined from the labor resource constraint $l(\lambda_L) = L/K$. To see that the solution to \eqref{statRCpsi} is unique whenever it exists, note that dividing by $\overline{C}(S)$ gives
\begin{equation}
1-\psi + \psi/\overline{C}(S) = {\left(S\sqrt{\rho}\phi \sigma/\alpha + \rho_S/\alpha + (1/\alpha-1)\delta \right)}(1-\psi)\frac{x(S, \overline{\omega})}{\sqrt{\rho}\phi \sigma} 
\label{statRCpsi2}
\end{equation}
and the right-hand side of \eqref{statRCpsi2} is increasing in $S$ while the left-hand side is decreasing in $S$. Finally, given $K$, the lifetime utility $u_0$ satisfies 
\begin{equation}
u_0 = \frac{AK^{\alpha}L^{1-\alpha} - \delta K}{(1-\psi)\overline{C}(S(\lambda_L)) + \psi}
\label{u0}
\end{equation}
which completes the characterization.
\end{proof} 


\section{Decentralization} \label{APPdecent} \subsection{Discrete-time approximation}

In order to aid the reader in this section I outline a discrete-time environment that approximates the incomplete-markets model of Section \ref{DecenGenEq}. Suppose that time assumes the values $\Delta, 2\Delta, \dots$ for some $\Delta >0$. The following is then the timing in the $n$th period (calendar time $t = n\Delta$). In the morning, the entrepreneur wakes up with $a_n$ units of wealth, consumes $\Delta c_n$ units of consumption, and deposits the remainder of their wealth in a bank that promises $(1+\Delta r)(a_n - \Delta c_n)$ tomorrow. In addition, the entrepreneur writes the following insurance contract against longevity risks of the following form: if alive tomorrow the insurance company will transfer $\Delta\rho_D(a_n - \Delta c_n)$ units of output to the entrepreneur, and otherwise the insurance company takes possession of the entrepreneur's wealth. In the afternoon, the entrepreneur supplies labor, rents $k_n$ units of capital from the bank on behalf of their firm, and invests the capital and hires workers at wage $w$. The output produced net of labor payments is $\Delta (\Pi + \delta) k_n$, and the depreciated capital is $(\delta \Delta + \sqrt{\Delta}\sigma X_n)k_n$, where $X_n$ is i.i.d. across both time and entrepreneurs and assumes the values $\pm 1$ with equal probability. The entrepreneur pays taxes $\tau_{sE}$ on savings (the sum of interest from the bank and the life insurance company) and $\tau_{\Pi}$ on profits, which are given by output minus labor costs, interest payments $\Delta rk_n$ on the borrowed capital, and the amount $s_nk_n\Delta$ stolen and diverted to the consumption
$$
\Delta (\Pi + \delta) k_n - (\delta \Delta + \sqrt{\Delta}\sigma X_n)k_n - \Delta r = {\left(\Delta (\Pi - r - s_n) - \sqrt{\Delta}\sigma X_n\right)} k_n.
$$
The wealth of the entrepreneur at the beginning of the next period is therefore
\begin{align*}
a_{n+1} & = (1+ \Delta (1-\tau_{sE}) (r+\rho_D)) (a_n - \Delta c_n) + \Delta (1-\tau_{LE})wL
\\ & + (1-\tau_{\Pi}){\left(\Delta (\Pi - r - s_n) - \sqrt{\Delta}\sigma X_n\right)} k_n
\end{align*}
and so the change in wealth satisfies
\begin{align*}
a_{n+1}-a_n & = -\Delta^2(1-\tau_{sE}) (r+\rho_D) c_n + {\left[ (1-\tau_{sE})(r+\rho_D)a_n - c_n + (1-\tau_{LE})wL + (1(\Pi - r)k_n\right]}\Delta 
\\ & + (1-\tau_{\Pi}){\left(\Delta (\Pi - r) - \sqrt{\Delta}\sigma X_n\right)} k_n.
\end{align*}
As $\Delta \rightarrow 0$ this becomes equivalent to the law given in Definition \ref{defnPROB}. The insurance company pays $\Delta \rho_D(a_n - \Delta c_n)$ with probability $1-\Delta \rho_D$ and receives $a_{n+1}$ with probability $\Delta \rho_D$, and so their expected profits are $-\Delta \rho_D(a_n - \Delta c_n)(1-\Delta \rho_D) + \Delta \rho_Da_n + \Delta \rho_D\mathbb{E}[a_{n+1} - a_n] $ or
\begin{align*}
\textnormal{Profits} & =  -\Delta \rho_D [(a_n - \Delta c_n)(1-\Delta \rho_D) - a_n
\\ & \ \ \  -[-\Delta(r+\rho_D) c_n + (r+\rho_D)a_n -  c_n + w + (\Pi - r)k_n]\Delta]
\\ & =  -\Delta \rho_D [a_n - a_n\Delta \rho_D - \Delta c_n + \Delta^2 c_n\rho_D - a_n
\\ & \ \ \  +[\Delta(r+\rho_D) c_n - (r+\rho_D)a_n +  c_n - w - (\Pi - r)k_n]\Delta]
\\ & = -\Delta^2 \rho_D [ - a_n \rho_D + \Delta c_n \rho_D + \Delta (r + \rho_D) c_n - (r + \rho_D) a_n - w - (\Pi - r)k_n]
\end{align*} %We have the standard labor market-clearing and goods market-clearing. 
which is $o(\Delta)$ as $\Delta \rightarrow 0$. Therefore, the profit over any fixed interval vanishes as $\Delta \rightarrow 0$. The bank makes zero profits, so the return promised to depositors is equal to the return demanded from entrepreneurs. The amount of wealth the agents deposit at the bank must equal the amount rented by entrepreneurs and so if the distribution of the wealth of entrepreneurs has density $g$ then market-clearing in the discrete-time environment is $\int_0^{\infty}(a - \Delta c(a))g(da) = \int_{0}^{\infty}k(a)g(da)$. Since the $\Delta c(a)$ term is negligible as $\Delta \rightarrow 0$, this approximates the market-clearing condition given in the proof of Proposition \ref{zerodriftEQ}. 

\subsection{Implementation with risk-free bond} \label{decentDT}

\begin{proof}[Proof of Lemma \ref{agentprobLEMMA}] \label{agentprobproof}
Given taxes on profits $\tau_{\Pi}$ and risk-free savings $\tau_{s}$, the Hamilton-Jacobi-Bellman equation for the agent's value function is 
\begin{align*}
\rho V(a) & = \max_{\substack{c,k \geq 0 \\ k \leq \hat{\omega}(a+h)}} \ \rho \ln c + {\left((1-\tau_s)(r+\rho_D)a - c + (1-\tau_{\Pi})(\Pi - r)k + (1-\tau_{LE})wL\right)}V'(a) 
\\ & + \frac{\sigma^2}{2} (1-\tau_{\Pi})^2k^2V''(a).
\end{align*}
Substituting $V(a) = \ln (a + h) + D$ in the Bellman equation and writing $\hat{c} = c/(a+h_E)$ and $\hat{k} = k/(a+h_E)$, we have 
\begin{align*}
\rho D & = \max_{\substack{\hat{c},\hat{k} \geq 0 \\ \hat{k} \leq \hat{\omega}}} \ \rho \ln \hat{c} + (1-\tau_s)(r+\rho_D) - \hat{c} + (1-\tau_{\Pi})(\Pi - r)\hat{k} - \frac{\sigma^2}{2} (1-\tau_{\Pi})^2\hat{k}^2
\end{align*}
which gives both claims upon substitution. 
\end{proof}

\begin{proof}[Proof of Proposition \ref{zerodriftEQ}]
The task of this proof is simply to show that for the above taxes and transfers, the conditions characterizing the efficient stationary distribution are satisfied. In this proof I will write $\hat{\mu}_{cE}$ and $\hat{\sigma}_{cE}$ for the coefficients of the drift and diffusion of entrepreneurs' consumption, and I will first show that $(\hat{\mu}_{cE}, \hat{\sigma}_{cE}) = (\mu_c(\hat{x}), \sqrt{\rho} \hat{x})$. For the above interest rate, profits tax, and constant in the collateral constraint, the capital policy function is
$$%\mu_c(\hat{x})/\rho = \max\{\ln(x/\overline{\omega}) + x^2/2, 0\}
\hat{k} = \frac{\sqrt{\rho}\hat{x}}{\phi\sigma}
$$%= \iota^{-1}e^{\mu_c(\hat{x})/\rho - \hat{x}^2/2}
because the inequality $\sqrt{\rho}\hat{x}/(\phi\sigma) \leq \hat{\omega} = \iota^{-1}\max\{\hat{x}/\overline{\omega}, e^{-\hat{x}^2/2}\}$ is automatically true. The consumption of entrepreneurs then satisfies $\hat{\mu}_{cE} = - \rho \hat{x}^2 + \mu_c(\hat{x}) + \phi \sigma \sqrt{\rho} \hat{x} \sqrt{\rho}\hat{x}/(\phi\sigma) = \mu_c(\hat{x})$ and $\hat{\sigma}_{cE} = (1-\tau_{\Pi})\sigma \sqrt{\rho}\hat{x}/(\phi\sigma) = \sqrt{\rho}\hat{x}$ while for workers we have $\hat{\mu}_{cW} = \hat{\sigma}_{cW} = 0$. For these coefficients, the constant appearing in the collateral constraint coincides with the general expression \eqref{hatomega}, and by Lemma \ref{agentprobLEMMA}, the entrepreneurs will be indifferent between revealing and not revealing their type when $\kappa_W = \kappa_E/\min\{e^{\hat{x}^2/2}, \overline{\omega}/\hat{x}\}$. It remains to verify that the market clearing conditions are satisfied at the marginal product of capital given in Proposition \ref{suffPROP}. The bond market clearing condition is 
\begin{align*}
K & = (1-\psi) \frac{\rho_D \kappa_E}{\rho_D - \mu_c(\hat{x})} \frac{\sqrt{\rho} \hat{x} K}{\phi\sigma}
\end{align*} 
which holds by the above definition of $\kappa_E$. Using $A(\hat{K}/L)^{\alpha-1} - \delta = \hat{\Pi}/\alpha + (1/\alpha-1)\delta$, then by cancelling $K$ on both sides, the goods market clearing equation becomes
\begin{align*}
(\hat{\Pi}/\alpha + (1/\alpha-1)\delta)(1-\psi) & = \rho{\left(\frac{\rho_D(1-\psi)\kappa_E}{\rho_D - \mu_c(\hat{x})} + \psi \kappa_W\right)} 
\\ & = \rho{\left(\frac{(1-\psi) \rho_D}{\rho_D - \mu_c(\hat{x})} + \psi \max\{e^{-\hat{x}^2/2}, \hat{x}/\overline{\omega}\}\right)} \frac{\phi \sigma (\rho_D - \mu_c(\hat{x}))}{\sqrt{\rho} \hat{x} \rho_D}
\end{align*} 
which coincides with \eqref{statRCpsi} and completes the proof. 
\end{proof} 

%The fact that $r < \rho_S$ follows from the above expression and Lemma \ref{Rfact}, and so i

\begin{proof}[Proof of Lemma \ref{negS}]
Note that for sufficiently small $\phi\sigma$, the no-absconding constraint holds as a strict inequality and the resource constraint becomes
\begin{equation}
(1-\psi)x(S, \overline{\omega}) = \frac{\alpha \sqrt{\rho}\phi \sigma(\psi/\overline{c}(S, \overline{\omega}) + 1-\psi)}{\rho_S + S\sqrt{\rho}\phi \sigma + (1-\alpha)\delta}.
\label{simpRC}
\end{equation} %$-\rho\hat{x}^2 > \hat{\Pi} - \rho_S - \sqrt{\rho} \sigma \hat{x}$
By Proposition \ref{zerodriftEQ}, the savings tax is negative if $\rho(1-\hat{x}^2) > \hat{\Pi} - \sqrt{\rho} \sigma \hat{x} + \rho_D$, or 
\begin{equation}
0 > \rho x(\hat{S},\overline{\omega})^2 + \hat{S}\sqrt{\rho}\phi \sigma - \sqrt{\rho} \sigma x(\hat{S},\overline{\omega}).
\label{negative}
\end{equation} %defining equation for $x$ is $Rx^2 - x + R = 0$, or $x^2 = x/R - 1$.
Using the inequalities in Lemma \ref{Rfact}, a sufficient condition for the tax to be negative is then $(1-\phi) \sigma/\sqrt{\rho} > 2\hat{x}$. Using \eqref{simpRC}, we then need
\begin{align*}
\frac{\alpha \sqrt{\rho}\phi \sigma(\psi/\overline{c}(S, \overline{\omega}) + 1-\psi)}{\rho_S + S\sqrt{\rho}\phi \sigma + (1-\alpha)\delta} & < (1-\psi)(1-\phi) \sigma/\sqrt{\rho},
\end{align*}
which holds for all sufficiently small $\phi$. If $\phi=1$, then the tax on savings will be positive if the right-hand side of \eqref{negative} is positive. Dividing by $\sqrt{\rho} \sigma x(\hat{S},\overline{\omega})$, this is equivalent to
\begin{equation}
1 < \sqrt{\rho} x(\hat{S},\overline{\omega})/\sigma + \hat{S}/x(\hat{S},\overline{\omega})
\label{smallsig}
\end{equation}
If $\sigma < \sqrt{\rho}$ and the no-absconding constraint does not hold as an equality, then it will suffice to show $x_{\textnormal{loc}}(S) < x_{\textnormal{loc}}(S)^2 + S$ for $S \in [0,1/2]$. Using $x_{\textnormal{loc}}(S) = (1 - \sqrt{1-4S^2})/[2S]$, this is equivalent to the inequalities
\begin{align*}
(1 - \sqrt{1-4S^2})/[2S] & < (1 - \sqrt{1-4S^2})^2/[4S^2] + S
%\\ 1 - \sqrt{1-4S^2} & < \frac{1}{2S}(1 - 2\sqrt{1-4S^2} + 1-4S^2) + 2S^2
\\ 1 - \sqrt{1-4S^2} & < \frac{1}{S}(1 - \sqrt{1-4S^2} - 2S^2) + 2S^2
\end{align*}
or $(1/S-1)\sqrt{1-4S^2} < (1 - 2S^2)(1/S - 1)$, which is true for all $S\in [0,1/2]$.
\end{proof} 

\section{Robustness and extensions} \label{robust_app}

In this section I discuss two extensions of the main result. 

\subsection{Heterogeneous entrepreneurs} \label{hetero_ent}

In this section I show how the characterization of efficient allocations and the above decentralization change in the presence of two types of entrepreneurs. Suppose that a fraction $\zeta \in [0,1]$ of the entrepreneurs operate with the technology $A_1K^{\alpha}L^{1-\alpha}$ and the remaining fraction operate with the technology $A_2K^{\alpha}L^{1-\alpha}$, and that the parameters governing the severity of the agency friction for each type are $\phi_1$ and $\phi_2$, respectively. The expressions $\Pi(\lambda_L), S(\lambda_L)$ and $l(\lambda_L)$ in Appendix \ref{STATapp} must now be indexed by $j \in \{1,2\}$,
\begin{align*}
l_j(\lambda_L) & = [(1-\alpha)A_j]^{1/\alpha}\lambda_L^{-1/\alpha}
\\ \Pi_j(\lambda_L) & = \alpha A_j^{1/\alpha} [(1-\alpha)/\lambda_L]^{1/\alpha-1} - \delta
\\ S_j(\lambda_L) & = \frac{\Pi_j(\lambda_L) - \rho_S}{\sqrt{\rho} \phi_j \sigma}.
\end{align*}
To characterize the new stationary efficient allocation, I want to write $\Pi_2$ and $S_2$ in terms of $\Pi_1$ and $S_1$. To this end, note that
\begin{equation}
\begin{aligned}
\Pi_2(\lambda_L) & = \alpha A_2^{1/\alpha} [(1-\alpha)/\lambda_L]^{1/\alpha-1} - \delta
\\ & = (A_2/A_1)^{1/\alpha} \Pi_1(\lambda_L) + ((A_2/A_1)^{1/\alpha}-1)\delta
=: D_0\Pi_1(\lambda_L) + D_1
\end{aligned}
\label{pinned}
\end{equation}
where the constants $D_0$ and $D_1$ do not depend on $\phi_1$ or $\phi_2$, and hence
\begin{align*} %= \frac{\Pi_2(\lambda_L) - \rho_S}{\sqrt{\rho} \phi_2 \sigma}
S_2(\lambda_L) & = \frac{D_0\Pi_1(\lambda_L) + D_1 - \rho_S}{\sqrt{\rho} \phi_2 \sigma}
%\\ & = \frac{D_0(\Pi_1(\lambda_L) - \rho_S) + D_1 + (D_0 - 1)\rho_S}{\sqrt{\rho} \phi_2 \sigma} 
 = \frac{\Pi_1(\lambda_L) - \rho_S}{\sqrt{\rho} \phi_1 \sigma} (D_0\phi_1/\phi_2) + \frac{D_1 + (D_0 - 1)\rho_S}{\sqrt{\rho} \phi_2 \sigma} 
\end{align*}
which is summarized as follows. 

\iffalse
\\ & = S_1(\lambda_L) (D_0\phi_1/\phi_2) + \frac{D_1 + (D_0 - 1)\rho_S}{\sqrt{\rho} \phi_2 \sigma} = S_1(\lambda_L) (A_2/A_1)^{1/\alpha}\phi_1/\phi_2 + ((A_2/A_1)^{1/\alpha}-1)\frac{\rho_S + \delta}{\sqrt{\rho} \phi_2 \sigma}
\fi

\begin{lemma}
Given a multiplier $\lambda_L$, we have 
\begin{equation}
\begin{aligned}
S_2(\lambda_L) & = S_1(\lambda_L) (A_2/A_1)^{1/\alpha}\phi_1/\phi_2 + ((A_2/A_1)^{1/\alpha}-1)(\rho_S + \delta)/(\sqrt{\rho} \phi_2 \sigma).
\label{S1S2}
\end{aligned}
\end{equation} %where $E_0 < 0$. 
\end{lemma} %\overline{\omega} is omitted from $\overline{c}$ and $x$. 

Note that $\overline{\omega}$ is common to both types and so I will drop it from the following notation for brevity. The characterization of the stationary efficient allocation now proceeds much as in the proof of Proposition \ref{statINEQ}, except that we have to be careful about non-negativity restrictions, as it might be the case the one type of entrepreneur is not producing in the stationary efficient allocation. The average consumption and capital delegated to entrepreneurs of each type in the stationary distribution per unit of initial utility are again given by the expressions \eqref{bigCK}, provided that we interpret $\overline{c}(S) = 1$ and $x(S) = 0$ for $S < 0$.

The planner once again places zero weight on the utility of all types of entrepreneurs, and so all agents receive the same level of initial utility $u_0$. Once again the variable that adjusts until resources clear is the multiplier $\lambda_L$ on the labor resource constraint, but it is convenient to write the resource constraint solely in terms of $S_1 = S$. Output in the stationary allocation for a fixed $\lambda_L$ and type $j \in \{0,1\}$ is
$$
J_j(S_j(\lambda_L)) := {\left(S_j(\lambda_L)/\alpha + (\rho_S/\alpha + (1/\alpha-1)\delta)/(\sqrt{\rho} \phi_j \sigma) \right)}\frac{\rho_D\overline{c}(S_j(\lambda_L))x(S_j(\lambda_L))}{\rho_D - \mu_c(S_j(\lambda_L))}.
$$ 
Now write $S_1 = S$ and $S_2(S) = S(A_2/A_1)^{1/\alpha}\phi_1/\phi_2 + ((A_2/A_1)^{1/\alpha}-1)(\rho_S + \delta)/(\sqrt{\rho}\phi_2\sigma)$, so that
\begin{align*}
J_1(S) & = {\left(S/\alpha + (\rho_S/\alpha + (1/\alpha - 1)\delta)/(\sqrt{\rho}\phi_1\sigma) \right)}\frac{\rho_D\overline{c}(S)x(S)}{\rho_D - \mu_c(S)} 
\\ J_2(S) & = {\left(S_2(S)/\alpha + (\rho_S/\alpha + (1/\alpha - 1)\delta)/(\sqrt{\rho} \phi_2 \sigma) \right)}
 \frac{\rho_D\overline{c}(S_2(S))x(S_2(S))}{\rho_D - \mu_c(S_2(S))}.
\end{align*}
Instead of \eqref{statRCpsi}, the equation characterizing the stationary distribution is now
\begin{equation}
(1-\psi)(\zeta \overline{C}(S) + (1-\zeta)\overline{C}(S_2(S))) + \psi = (1-\psi)(\zeta J_1(S) + (1 - \zeta) J_2(S)).
\label{statRCpsi3}
\end{equation}
The situation in which we add a small number of more productive entrepreneurs to the benchmark case corresponds to $\phi_1 = \phi_2$, $A_2 > A_1$ and $\zeta \rightarrow 1$, so that the resource constraint defining $\hat{S}$ is unaffected, provided that for these values, $S_2(\hat{S})$ satisfies Assumption \ref{ASSsuff} and Assumption \ref{ASSsuff2}.

%be careful that we still have $\iota_1 = \phi_1 \overline{\iota}$ and $\iota_2 = \phi_2 \overline{\iota}$. 
%do we get a discontinuity here? Think slowly. Let $\Pi$ be the 

For the second example, suppose that type 1 entrepreneurs are not subject to an agency problem, so that $\phi_1 \approx 0$, and that type 2 entrepreneurs are more productive, so that $A_2 > A_1$. In this case, the sole effect of the corporate sector is to put an upper bound on the marginal product of capital that can obtain in the efficient allocation. Specifically, the multiplier $\lambda_L$ cannot be any smaller than the value at which $\Pi_1(\lambda_L) = \rho_S$, and so we must have 
$$
\Pi_2 - \rho_S = \min\{\hat{\Pi}_2 - \rho_S, ((A_2/A_1)^{1/\alpha} - 1)(\rho_S + \delta)\},
$$
where $\hat{\Pi}_2$ is the efficient level of the marginal product of capital is a world with only type 2 entrepreneurs. To understand this, note that when the corporate sector is utilized, the marginal product of capital in this sector is pinned down at $\rho_S$, and the associated multiplier on the resource constraint satisfies $\alpha A_1^{1/\alpha} [(1-\alpha)/\lambda_L^*]^{1/\alpha-1} = \rho_S + \delta$ or 
$$
\lambda_L^* = A_1^{\frac{1}{1-\alpha}}(\alpha/(\rho_S + \delta))^{\frac{\alpha}{1-\alpha}}(1-\alpha).
$$
From \eqref{pinned}, this then determines the marginal product of capital among the entrepreneurs at $\Pi_2(\lambda_L^*)$. When the no-absconding constraint does not bind, the fraction of output that is produced by the entrepreneurial sector is the solution $\xi$ to 
$$
\xi(1-\psi + \psi/\overline{C}(S_2(\lambda_L^*))) = {\left((\rho_S + S_2(\lambda_L^*)\sqrt{\rho}\phi \sigma)/\alpha + (1/\alpha-1)\delta \right)}(1-\psi)\frac{x(S_2(\lambda_L^*),\overline{\omega})}{\sqrt{\rho}\phi \sigma}.
$$

\subsection{Constant relative risk aversion}\label{CRRA}

In this section I show that the qualitative claims in the main text extend to the case of constant relative risk aversion with $\gamma \geq 1$. I will not strive for complete generality, as I wish to show only that the method followed in the main text does not depend crucially on logarithmic utility. In particular, when the no-absconding does not hold with equality, the inverse Euler equation holds and the constrained-efficient allocation can be decentralized by choosing taxes and transfers to match the constant mean and variance of consumption growth in the efficient allocation. However, it is more difficult to provide sufficient conditions for the principal's problem to be well-defined and for the no-absconding constraint to not bind in the optimal contract. 

\iffalse
The main simplifications afforded by logarithmic utility are:
\begin{itemize}
\item in the efficient allocation: $x_{\textnormal{loc}}$ in \eqref{kcx} has a simple solution; 
\item in the decentralization: income and substitution effects cancel, and consumption is a constant fraction of lifetime wealth.
\end{itemize}
However, the basic approach remains valid in the more general case, with only the algebra becoming more complicated.
\begin{align*}
\frac{\overline{c}^{1-\gamma} - \gamma}{1-\gamma} - \gamma z^2\overline{c}^{2-2\gamma}/2 & = \overline{c}^{1-\gamma} - \gamma(1-\gamma) z^2\overline{c}^{2-2\gamma}
%\\ \frac{\overline{c}^{1-\gamma} - 1}{1-\gamma} & = (\gamma-1/2) z^2\overline{c}^{2-2\gamma}
\\ 0 & = 1 - \overline{c}^{1-\gamma} + (1-\gamma)(\gamma-1/2) z^2\overline{c}^{2-2\gamma}
\end{align*}
\fi

\subsubsection{Characterizaton of value function}

Proceeding analogously as in Appendix \ref{agency_IC}, when characterizing efficient allocations with utility function $u(c) = c^{1-\gamma}/(1-\gamma)$, it is without loss to assume that promised utility satisfies
$$
dW_t = \rho{\left(W_t - c_t^{1-\gamma}/(1 - \gamma) \right)}dt + \rho \phi \sigma k_t c_t^{-\gamma} dB_t. 
$$
Now define utility in consumption units as $u_t = [(1-\gamma)W_t]^{\frac{1}{1-\gamma}}$ and change variables to $c_t = \overline{c}_t[(1-\gamma)W_t]^{\frac{1}{1-\gamma}}$ and $k_t = \overline{k}_t[(1-\gamma)W_t]^{\frac{1}{1-\gamma}}$. The law of motion of promised utility is then 
$$
dW_t = \rho{\left(1 - \overline{c}_t^{1-\gamma}\right)}W_t dt + \rho \phi \sigma \overline{k}_t \overline{c}_t^{-\gamma} (1-\gamma)W_t dB_t =: \mu_WW_tdt + \sigma_WW_t dB_t.
$$
If $f(W_t) := [(1-\gamma)W_t]^{\frac{1}{1-\gamma}}$ then $f'(W_t) := [(1-\gamma)W_t]^{\frac{\gamma}{1-\gamma}}$ and $f''(W_t) := \gamma[(1-\gamma)W_t]^{\frac{\gamma}{1-\gamma}-1}$, and so using Ito's lemma the law of motion for $u_t$ as 
\begin{align*}
df(W_t) & = (\mu_WW_t f'(W_t)+ (\sigma_W^2/2)W_t^2f''(W_t))dt + f'(W_t)\sigma_WW_tdB_t
\\ & = \rho {\left( \frac{(1 - \overline{c}_t^{1-\gamma})}{1-\gamma} [(1-\gamma)W_t] f'(W_t) + (\sqrt{\rho} \phi \sigma \overline{k}_t \overline{c}_t^{-\gamma})^2[(1-\gamma)W_t]^2f''(W_t)/2\right)}dt 
\\ & + \rho \phi \sigma \overline{k}_t \overline{c}_t^{-\gamma}[(1-\gamma)W_t] f'(W_t)dB_t
\end{align*}
which may be written as 
$$
du_t = \rho{\left( \frac{1 - \overline{c}_t^{1-\gamma}}{1-\gamma} + \gamma x_t^2/2\right)}u_t dt + \sqrt{\rho} x_t u_t dB_t
$$
where $x := \sqrt{\rho} \phi \sigma k c^{-\gamma}$. As with Section \ref{PAmodel}, the value function is linear in $u_t$ whenever it is finite-valued, and so it is of the form $v(u) = \overline{v}u$, where
\begin{equation}
\overline{v} = \sup_{\substack{\overline{c},x\geq0, x\overline{c}^{\gamma} \leq \overline{\omega} \\ -1 < \frac{\overline{c}^{1-\gamma} - 1}{1-\gamma} - \frac{\gamma x^2}{2}}} \frac{(Sx\overline{c}^{\gamma} - \overline{c})/\rho}{1 + (\overline{c}^{1-\gamma}-1)/(1-\gamma) - \gamma x^2/2}
\label{cand2}
\end{equation} %$Sx(\overline{\omega}/x)^{1-1/\gamma} > 1$. $\overline{x}$ and 
is a candidate value function. Now define $\overline{\overline{x}}$ to be the solution to
\begin{align*}
%\frac{\gamma \overline{x}^2}{2} + \frac{1 - (\overline{\omega}/\overline{x})^{1/\gamma-1}}{1-\gamma} & = 0
\frac{\gamma \overline{\overline{x}}^2}{2} + \frac{1 - (\overline{\omega}/\overline{\overline{x}})^{1/\gamma-1}}{1-\gamma} & = 1
\end{align*}
and consider the following two analogues of Assumptions \ref{ASSsuff} and \ref{ASSsuff2}. 

\begin{assump}\label{ASSsuffCRRA}
$S\overline{\overline{x}}^{1/\gamma}\overline{\omega}^{1-1/\gamma} < 1$. 
\end{assump}

\begin{assump}\label{ASSsuffCRRA2}
$\overline{\omega}^{1/\gamma}S^{1/\gamma} + \rho{\left(\overline{\omega}^{1/\gamma-1}S^{1/\gamma-1} + \gamma^2 S^{-2}\right)}\overline{v} \leq 0$.
\end{assump}

Assumptions \ref{ASSsuffCRRA} and \ref{ASSsuffCRRA2} are obviously satisfied for all sufficiently small $S$. The following establishes the main claims of Proposition \ref{existence} and Proposition \ref{suffLEM} for the case of CRRA preferences. 

%0 = 1 + ((\overline{\omega}/z)^{1-\gamma}-1)/(1-\gamma) - \gamma z^{2\gamma}\overline{\omega}^{2-2\gamma}/2.
%he optimal choice of $x$ is increasing in $S$ wherever it is well-defined, and t
\begin{prop} \label{existenceCRRA}
If Assumptions \ref{ASSsuffCRRA} and \ref{ASSsuffCRRA2} are satisfied, then the value function is finite-valued and given by $v(u) = \overline{v}u$ for all $u > 0$. Further, the no-absconding constraint holds for sufficiently small $S$, in which case the inverse Euler equation holds. 
\end{prop}

\begin{proof}
I first show that $\overline{v} \leq 0$ for sufficiently small $S$. Note that because $\gamma \geq 1$, if there existed a pair $(x, \overline{c})$ in the constraint set with $Sx\overline{c}^{\gamma-1} > 1$, then there would also exist such a pair in which the no-absconding constraint held with equality $\overline{c} = (\overline{\omega}/x)^{1/\gamma}$. i.e. there would exist $x$ such that $Sx^{1/\gamma}\overline{\omega}^{1-1/\gamma} > 1$ and 
$$
\frac{\gamma x^2}{2} + \frac{1 - (\overline{\omega}/x)^{1/\gamma-1}}{1-\gamma} < 1
$$
which violates Assumption \ref{ASSsuffCRRA}. As per the proof of Proposition \ref{existence}, to establish that the optimal choices in \eqref{cand2} solve the Hamilton-Jacobi-Bellman equation
$$
0 = \sup_{\substack{\overline{c},x\geq 0 \\ x\overline{c}^{\gamma} \leq \overline{\omega}}} (Sx\overline{c}^{\gamma-1} - 1)\overline{c} + \rho{\left(\frac{1 - \overline{c}^{1-\gamma}}{1-\gamma} + \gamma x^2/2 - 1\right)}\overline{v} =: \sup_{\substack{\overline{c},x\geq 0 \\ x\overline{c}^{\gamma} \leq \overline{\omega}}} H(\overline{c}, x, \overline{v}),
$$
it remains to eliminate the possibility that $H(\overline{c}, x, \overline{v}) > 0$ for some $(\overline{c}, x)$ satisfying $\overline{c},x\geq 0$ and $x\overline{c}^{\gamma} \leq \overline{\omega}$. The existence of $(\overline{c}, x)$ satisfying $(1 - \overline{c}^{1-\gamma})/(1-\gamma) + \gamma x^2/2- 1 < 0$ would violate the definition of $\overline{v}$, and so it will suffice to rule out the existence of such a pair $(Sx\overline{c}^{\gamma-1} - 1)\overline{c} + \rho((1 - \overline{c}^{1-\gamma})/(1-\gamma) + \gamma x^2/2 - 1)\overline{v} > 0$. Since $\overline{v} < 0$, this last inequality implies %requires $Sx\overline{c}^{\gamma-1} > 1$, and so would
\begin{equation} %S\overline{\omega} - (\overline{\omega}/x)^{1/\gamma}
(Sx^{1/\gamma}\overline{\omega}^{1-1/\gamma} - 1)(\overline{\omega}/x)^{1/\gamma} + \rho{\left(\frac{1 - (\overline{\omega}/x)^{1/\gamma-1}}{1-\gamma} + \gamma x^2/2 - 1\right)}\overline{v} > 0
\label{impossible}
\end{equation}
for some $x > \overline{\overline{x}}$. The inequality \eqref{impossible} cannot hold on $x \in [\overline{\overline{x}}, 1/S]$ from Assumption \ref{ASSsuffCRRA}, and so to establish finiteness it will suffice for the derivative of the left-hand side to be negative for all $x \geq 1/S$. This is equivalent to 
$$
\overline{\omega}^{1/\gamma}x^{-1/\gamma} + \rho{\left(\overline{\omega}^{1/\gamma-1}x^{1-1/\gamma} + \gamma^2 x^2\right)}\overline{v} \leq 0.
$$
Because $\overline{v}<0$, this will be true for $x \geq 1/S$ if and only if it is true for $x=1/S$, which is exactly Assumption \ref{ASSsuffCRRA2}. 

Ito's Lemma implies that the inverse Euler equation is equivalent to $\mu_c = {\left(1 - \gamma\right)}\sigma_c^2/2$. Whenever the value function is finite, the optimal $\overline{c}$ and $x$ in \eqref{cand2} minimize
\begin{align*}
\ln (\overline{c} - Sx\overline{c}^{\gamma}) - \ln {\left(\frac{\overline{c}^{1-\gamma} - \gamma}{1-\gamma} - \gamma x^2/2 \right)}.
\end{align*} %(and hence $x=z\overline{c}^{1-\gamma}$) 
Now suppose that we fix $z := x\overline{c}^{\gamma-1}$ and minimize over $\overline{c}$. This gives
\begin{equation}
\ln (1 - Sz) + \min_{\substack{\overline{c} > 0, \overline{c}z \leq \overline{\omega} \\ \frac{\overline{c}^{1-\gamma} - \gamma}{1-\gamma} > \gamma x^2/2}} \ln \overline{c} - \ln {\left(\frac{\overline{c}^{1-\gamma} - \gamma}{1-\gamma} - \gamma z^2\overline{c}^{2-2\gamma}/2 \right)}.
\label{minoverc}
\end{equation}
Now write $y := \overline{c}^{1-\gamma}$, so that the problem is equivalent to maximizing
\begin{align*}
\ln y + (\gamma - 1) \ln {\left(\frac{y - \gamma}{1 - \gamma} - \gamma z^2y^2/2 \right)}
\end{align*}
over the set of $y > 0$ such that $y \geq (\overline{\omega}/z)^{\frac{1}{1-\gamma}}$ and $(y - \gamma)/(1 - \gamma) - \gamma z^2y^2/2 > 0$. This last function is concave and diverges when $(y - \gamma)/(1 - \gamma) \approx \gamma z^2y^2/2 > 0$, and so the optimal choice of consumption in \eqref{minoverc} is the minimum of the solution to the first-order condition and the boundary value $\overline{c} = \overline{\omega}/z$. The first-order condition for consumption is
\begin{align*}
\frac{1}{\overline{c}} & = \frac{\overline{c}^{-\gamma} - \gamma(1 - \gamma) z^2\overline{c}^{1 - 2\gamma}}{\frac{\overline{c}^{1-\gamma} - \gamma}{1 - \gamma} - \gamma z^2\overline{c}^{2 - 2\gamma}/2}.
\end{align*}
Rearranging then gives a quadratic in $\overline{c}^{1-\gamma}$, $0 = 1 - \overline{c}^{1-\gamma} + (1-\gamma)(\gamma-1/2) z^2\overline{c}^{2-2\gamma}$, which has one positive solution for consumption which I denote by $\overline{c}_{\textnormal{foc}}(z)$. Note that $\overline{c}_{\textnormal{foc}}(z)$ is necessarily increasing in $z$. Simplifying and using the definition of $x$ gives
\begin{align*}
\frac{\overline{c}_{\textnormal{foc}}^{1 - \gamma} - \gamma}{1 - \gamma} - \gamma x^2/2 & = \overline{c}_{\textnormal{foc}}^{1 - \gamma} - \gamma (1 - \gamma) x^2,
\end{align*}
and hence $(\overline{c}_{\textnormal{foc}}^{1-\gamma} - 1)/(1 - \gamma) = - (1/2 - \gamma) x^2$, which is equivalent to the inverse Euler equation. In what follows I write $\overline{c}(z) = \min\{\overline{c}_{\textnormal{foc}}(z), \overline{\omega}/z\}$. Now write $\overline{z}$ for the solution to $\overline{c}_{\textnormal{foc}}(\overline{z}) = \overline{\omega}/\overline{z}$, which occurs when 
$$
0=1-(\overline{\omega}/\overline{z})^{1-\gamma}+(1-\gamma)(\gamma-1/2)\overline{\omega}^{2-2\gamma}\overline{z}^{2\gamma}. 
$$
The no-absconding constraint will hold as a strict inequality if the optimal $z$ lies in $[0, \overline{z}]$. 
\end{proof}

\iffalse
It remains to show that this holds for sufficiently small $S$. The optimal $z$ then solves
\begin{align*}
\sup_{z \in [0, \overline{z}]} \frac{(Sz - 1)\overline{c}(z)}{1 + (\overline{c}(z)^{1-\gamma}-1)/(1-\gamma) - \gamma z^2\overline{c}(z)^{2-2\gamma}/2}.
\end{align*}
It will suffice to show that for sufficiently small $S$, $z=0$ gives a higher payoff than anything above $\overline{z}$. Can we make this uniform? For $z \geq \overline{z}$, we have 
\begin{align*}
\frac{S\overline{\omega} - \overline{\omega}/z}{1 + ((\overline{\omega}/z)^{1-\gamma}-1)/(1-\gamma) - \gamma z^{2\gamma}\overline{\omega}^{2-2\gamma}/2}
\end{align*}
\fi


\iffalse

I want to show that the optimal choice of $z(S, \overline{\omega})$ is increasing in $S$. From Topkis' theorem, we want 
$$
\frac{z\overline{c}(z)}{1 + (\overline{c}(z)^{1-\gamma}-1)/(1-\gamma) - \gamma z^2\overline{c}(z)^{2-2\gamma}/2}
$$
is increasing in $z$. For $z > \overline{z}$, this is 
$$
\frac{\overline{\omega}}{1 + ((\overline{\omega}/z)^{1 - \gamma} - 1)/(1-\gamma) - \gamma \overline{\omega}^{2-2\gamma}z^{2\gamma}/2}
$$
This is increasing in $z$. What about on $[0, \overline{z}]$? Using the definition of $\overline{c}_{\textnormal{foc}}(z)$, this gives
$$
\frac{\overline{c}_{\textnormal{foc}}^{1-\gamma} - 1}{1-\gamma} = (\gamma-1/2) z^2\overline{c}_{\textnormal{foc}}^{2-2\gamma}
$$
I want to show that the following is increasing in $z$,
\begin{align*}
& \frac{z\overline{c}_{\textnormal{foc}}(z)}{1 + \frac{\overline{c}_{\textnormal{foc}}(z)^{1-\gamma} - 1}{1-\gamma} - \frac{\overline{c}_{\textnormal{foc}}(z)^{1-\gamma} - 1}{1-\gamma}(\gamma/2)/(\gamma-1/2)}
\end{align*}
The denominator becomes
\begin{align*}
 1 + \frac{\overline{c}_{\textnormal{foc}}(z)^{1-\gamma}-1}{1-\gamma}{\left(1 - \frac{\gamma /2}{\gamma-1/2}\right)}
 & = 1 + \frac{\overline{c}_{\textnormal{foc}}(z)^{1-\gamma}-1}{1-\gamma}{\left(\frac{\gamma - 1}{2\gamma-1}\right)}
\\ & = 1 + \frac{1}{2\gamma-1} + \frac{\overline{c}_{\textnormal{foc}}(z)^{1-\gamma}}{1-\gamma}{\left(\frac{\gamma - 1}{2\gamma-1}\right)}
\\ & = \frac{2\gamma - \overline{c}_{\textnormal{foc}}(z)^{1-\gamma}}{2\gamma-1}.
\end{align*}
I then want the following to be increasing in $z$
$$
\frac{z\overline{c}_{\textnormal{foc}}(z)}{2\gamma - \overline{c}_{\textnormal{foc}}(z)^{1-\gamma}}
$$



I may need to perform an explicit maximization here. The solution $\overline{c}_{\textnormal{foc}}(z)$ is given by 
$$ %0 = -1 + \overline{c}^{1-\gamma} + (\gamma-1)(\gamma-1/2) z^2\overline{c}^{2-2\gamma}
\overline{c}^{1-\gamma} = \frac{1}{(\gamma-1)(2\gamma-1) z^2} {\left(-1 + \sqrt{1 + 2(\gamma-1)(2\gamma-1) z^2} \right)}
$$
It will certainly suffice to have the following increasing.
$$
\frac{1}{2\gamma - \frac{1}{(\gamma-1)(2\gamma-1) z^2} {\left(-1 + \sqrt{1 + 2(\gamma-1)(2\gamma-1) z^2} \right)}} {\left(\sqrt{1 + 2(\gamma-1)(2\gamma-1) z^2} - 1 \right)}/z.
$$
Rearranging gives 
\begin{align*}
\frac{(\gamma-1)(2\gamma-1) z}{2\gamma(\gamma-1)(2\gamma-1)z^2 + 1 - \sqrt{1 + 2(\gamma-1)(2\gamma-1) z^2}} {\left(\sqrt{1 + 2(\gamma-1)(2\gamma-1) z^2} - 1 \right)}.
\end{align*}

Can I obtain the derivative of $c(z)$ with respect to 
I want to show that the following is increasing in $z$, 
$$
\frac{z\overline{c}_{\textnormal{foc}}(z)}{1 + (\gamma - 1) z^2\overline{c}_{\textnormal{foc}}(z)^{2-2\gamma}/2}.
$$
This is subtle, because the numerator and denominator are both increasing. However, write this as 
$$
\frac{z\overline{c}_{\textnormal{foc}}(z)^{1-\gamma}\overline{c}_{\textnormal{foc}}(z)^{\gamma}}{1 + (\gamma - 1) z^2\overline{c}_{\textnormal{foc}}(z)^{2-2\gamma}/2}.
$$
\fi


\iffalse
Now write $\overline{\overline{z}}$ for the solution to 
$$
1 + \frac{(\overline{\omega}/z)^{1-\gamma}-1}{1-\gamma} - \frac{\gamma}{2} z^2(\overline{\omega}/z)^{2-2\gamma} = 0. 
$$
\fi

%recall $z := x\overline{c}^{\gamma-1}$

\iffalse
I write the policy functions in \eqref{cand2} as $\overline{c}(S, \overline{\omega})$ and $x(S, \overline{\omega})$, and write the drift in consumption as
\begin{equation} 
\mu_c(S,\overline{\omega}) := \rho {\left(\frac{1-\overline{c}(S, \overline{\omega})^{1-\gamma}}{1-\gamma} + \gamma x(S, \overline{\omega})^2/2\right)}.
\label{muCRRA}
\end{equation}
\fi

\subsubsection{Efficient stationary distribution and decentralization} \label{decent}

For general CRRA parameters, the Hamilton-Jacobi-Bellman equation is given by
\begin{align*}
\rho V(a) & = \max_{\substack{c, k \geq 0 \\ k \leq \hat{\omega}(a+h_E) } } \ \frac{c^{1-\gamma}}{1-\gamma} + {\left[(1-\tau_{sE})(r + \rho_D)a - c + (1 - \tau_{\Pi})(\Pi - r)k + (1-\tau_{LE})wL\right]}V'(a)
\\ & + (1-\tau_{\Pi})^2\frac{\sigma^2k^2}{2} V''(a) .
\end{align*}

\begin{lemma}\label{CRRAap}
The entrepreneur choose not to divert if and only if $\tau_{\Pi} \leq 1 - \phi$. In this case the value function is of the form $V(a) = \overline{V}(a+h_E)^{1-\gamma}/(1-\gamma)$ for some $\overline{V}$, and
\begin{align*} 
c(a) & = \hat{c}(a+h_E) = {\left(\frac{1}{\gamma} [\rho -  (1-\gamma)(1-\tau_{sE})(r+\rho_D)] - \frac{(\Pi - r)^2}{2\gamma^2\sigma^2} (1-\gamma)\right)}(a+h_E)
\\ k(a) & = \hat{k}(a+h_E) = \frac{(\Pi-r)(a+h_E)}{\gamma \sigma^2(1-\tau_{\Pi})}.
\end{align*}
The associated law of motion of wealth is $da = \mu_a(a_t+h_E)dt + \sigma_a(a_t+h_E)dZ_t$, where
\begin{align*}
\mu_a & = \frac{1}{\gamma}[(1 - \tau_{sE})(r + \rho_D) - \rho] + \frac{(\Pi - r)^2}{2\gamma^2\sigma^2}(1+\gamma)  &
\sigma_a & = \frac{\Pi - r}{\gamma \sigma}.
\end{align*}
\end{lemma}

\begin{proof}
Upon substituting the assumed form, the Hamilton-Jacobi-Bellman equation becomes 
\begin{align*}
\frac{\rho \overline{V}}{1-\gamma} & = \max_{\hat{c}, \hat{k}} \ \frac{\hat{c}^{1-\gamma}}{1-\gamma} + \overline{V}{\left[(1-\tau_{sE})(r+\rho_D) - \hat{c} + (1-\tau_{\Pi})(\Pi - r)\hat{k}\right]} - \gamma(1-\tau_{\Pi})^2\overline{V}\frac{\sigma^2\hat{k}^2}{2}.
\end{align*}
First-order conditions for capital and consumption give
\begin{align*}
\hat{k} & = \frac{\Pi - r}{ \gamma (1-\tau_{\Pi})\sigma^2} &
\hat{c} & = \overline{V}^{-1/\gamma}.
\end{align*}
Substituting into the Hamilton-Jacobi-Bellman equation gives 
\begin{align*}
\frac{\rho \overline{V}}{1-\gamma} & =  \frac{\overline{V}^{1-1/\gamma} }{1-\gamma} + \overline{V}{\left[(1-\tau_{sE})(r + \rho_D) - \hat{c} \right]} + \overline{V}{\left[ (1-\tau_{\Pi})(\Pi - r)\hat{k} - \frac{\gamma }{2} [(1-\tau_{\Pi})\sigma]^2\overline{k}^2\right]}
\\  & =  \frac{\gamma\overline{V}^{1-1/\gamma}}{1-\gamma} + \overline{V} (1-\tau_{sE})(r + \rho_D) + \overline{V}{\left[\frac{(\Pi-r)^2}{\gamma\sigma^2} - \frac{\gamma }{2} [(1-\tau_{\Pi})\sigma]^2{\left(\frac{\Pi-r}{\gamma (1-\tau_{\Pi})\sigma^2}\right)}^2\right]}
\\ \frac{\rho }{1-\gamma} & = \frac{\gamma\overline{V}^{-1/\gamma}}{1-\gamma} + (1-\tau_{sE})(r + \rho_D) + \frac{{\left(\Pi-r \right)}^2}{2\gamma\sigma^2} 
\end{align*}
which rearranges to 
\begin{align*}
\hat{c} = \overline{V}^{-1/\gamma} & = \frac{1}{\gamma} [\rho -  (1-\gamma)(1-\tau_{sE})(r+\rho_D)] - \frac{{\left(\Pi - r\right)}^2}{2\gamma^2\sigma^2} (1-\gamma)
\end{align*}
as claimed. The law of motion of wealth is then 
\begin{align*}
da_t & = {\left[(1-\tau_{sE})(r + \rho_D)a_t + (1-\tau_{LE})wL - c_t + (1 - \tau_{\Pi})(\Pi-r)k_t\right]}dt + (1-\tau_{\Pi})\sigma k_t dZ_t
\end{align*}
and so the law of total wealth is
\begin{align*}
\frac{d(a_t + h_E)}{(a_t + h_E)} & = {\left[(1-\tau_{sE})(r + \rho_D) - \overline{c} + (1-\tau_{\Pi})(\Pi - r)\hat{k}\right]}dt + (1-\tau_{\Pi})\sigma \hat{k} dZ_t
\\ & = {\left[(1-\tau_{sE})(r + \rho_D) - \frac{1}{\gamma} [\rho - (1-\gamma)(1-\tau_{sE})(r+\rho_D)] + \frac{{\left(\Pi-r\right)}^2}{2\gamma^2\sigma^2} (1-\gamma) + \frac{(\Pi-r)^2}{\gamma \sigma^2}\right]}dt 
\\ & + \frac{(\Pi-r)}{\gamma\sigma}dZ_t.
\end{align*}
This implies $\sigma_a = (\Pi - r)/(\gamma \sigma)$, while $\mu_a$ simplifies to 
\begin{align*}
\mu_a & =  - \frac{\rho}{\gamma} + \frac{1}{\gamma}(1-\tau_{sE})(r+\rho_D) + \frac{{\left(\Pi - r\right)}^2}{\gamma\sigma^2}{\left( \frac{1}{2}(1/\gamma-1) +  1\right)}
\end{align*}
as claimed.
\end{proof}

The efficient allocation is again characterized by three properties: the marginal product of capital coincides with the solution to the stationary form of the goods resource constraint, and the mean and volatility of the growth in consumption coincide with those in the efficient allocation. For the latter, we require that the mean and volatility of consumption growth be $\mu_C = \rho(1-\gamma)x^2/2$ and $\sigma_C = \sqrt{\rho} x$. 

\begin{prop} \label{propCRRA}
The marginal product of capital that obtains in the efficient stationary distribution is $\hat{\Pi} = \rho_S + \hat{S} \sqrt{\rho} \phi \sigma$, where $\hat{S}$ is the solution to the equation
\begin{equation}
\frac{\rho_D\overline{c}(S, \overline{\omega})}{\rho_D - \mu_c(S, \overline{\omega})} + \frac{\psi}{1-\psi} = {\left[S\sqrt{\rho} \phi \sigma/\alpha + \rho_S/\alpha + (1/\alpha-1)\delta \right]} \frac{\rho_D \overline{c}(S, \overline{\omega}) x(S, \overline{\omega})}{(\rho_D - \mu_c(S, \overline{\omega}))\sqrt{\rho}\phi \sigma}
\label{STATpi}
\end{equation}
provided that $\mu_c(\hat{S}, \overline{\omega}) < \rho_D$. In this case, writing $\hat{x} = x(\hat{S},\overline{\omega})$, the efficient allocations may be decentralized in a competitive equilibrium in which $\tau_{\Pi} = 1-\phi$, the real interest rate is $r = \hat{\Pi} - \gamma \sqrt{\rho} \sigma \hat{x}$, the tax on profits is $\tau_{\Pi} = 1 - \phi$, and the taxes on savings are
\begin{align*}
1 - \tau_{sE} & = \frac{\rho(1 - \gamma^2\hat{x}^2) + \mu_c(\hat{x})}{r + \rho_D} &
1 - \tau_{sW} & = \frac{\rho}{r + \rho_D}.
\end{align*}
The endowed wealth of entrepreneurs as a fraction of the capital stock is
\begin{align*}
\kappa_E & = \frac{\phi \sigma (\rho_D - \mu_c(\hat{x}))}{\sqrt{\rho} \hat{x} (1-\psi) \rho_D} 
\end{align*}
and $\kappa_W$ is chosen such that entrepreneurs and workers obtain the same level of lifetime utility.
\end{prop}

\iffalse
\mu_a & = \frac{1}{\gamma}[(1 - \tau_{sE})(r + \rho_D) - \rho] +  \frac{(\Pi - r)^2}{2\gamma^2\sigma^2}(1+\gamma)  &
\sigma_a & = \frac{\Pi - r}{\gamma \sigma}. \fi

\begin{proof}%[Proof of Proposition \ref{propCRRA}]
The proof proceeds in much the same way as the proof of Proposition \ref{zerodriftEQ}. The risk borne by the agent when $\tau_{\Pi} = 1-\phi$ is $\phi \sigma \hat{k}$, which from Lemma \ref{CRRAap} implies that the entrepreneurs bear the efficient level of risk if $\sqrt{\rho} \hat{x} = (\hat{\Pi} - r)/(\gamma \sigma)$, which is true for the above choice of interest rate. In this case, the mean growth in entrepreneurs' consumption is
\begin{align*}
\hat{\mu}_{cE} & = \frac{1}{\gamma}(r_E - \rho) + \frac{(\hat{\Pi} - r)^2}{2\gamma^2\sigma^2}(1+\gamma) 
 = \frac{1}{\gamma}(r_E - \rho) + \rho \hat{x}^2(1+\gamma)/2.
\end{align*}
In order for the inverse Euler equation to hold, we need $\hat{\mu}_{cE} = \rho(1-\gamma)\hat{x}^2/2$, which requires $\rho(1-\gamma)\hat{x}^2/2 = (r_E - \rho)/\gamma + \rho \hat{x}^2(1+\gamma)/2$ and hence $r_E = \rho(1 - \gamma^2 \hat{x}^2)$ which holds for the above tax on savings. 
\end{proof}

\section{Additional figures} \label{figuresAPP}

In this appendix I depict some additional figures related to the examples computed in Section \ref{numerical}. Throughout, parameters remain fixed at \eqref{standardPARAMS}.

\begin{figure}[H]
\centering
\includegraphics[width=0.49\linewidth, height=0.35\linewidth]{nu_B}
\includegraphics[width=0.49\linewidth, height=0.35\linewidth]{nu_K}
%\includegraphics[width=0.49\linewidth, height=0.35\linewidth]{nu_B_minus_n_K}
\caption{Wedges on entrepreneurs}
\label{wedge} 
\end{figure} 

Figure \ref{wedge} depicts the wedges faced by entrepreneurs, which are simply properties of the efficient allocation and do not depend on the market structure. As expected from Proposition \ref{suffPROP}, the wedge on the risk-free asset is positive and increasing in agency frictions. In contrast, the wedge on the risky asset exhibits no such monotonicity. 

In contrast with the main text, in this appendix I consider the market structure with private equity contracts discussed in Section \ref{discuss}. Figure \ref{tax_pe} and Figure \ref{tax_pe1.0} depict the taxes on savings, while Figure \ref{r_pe} depicts the real interest rate. In all cases I show the results for both the tight $(\overline{\iota} = 1.0)$ and relaxed $(\overline{\iota} = 0.5)$ collateral constraints. 

\begin{figure}[H]
\centering
\includegraphics[width=0.49\linewidth, height=0.35\linewidth]{taxes_entrepreneurs_pe1.0}
%\includegraphics[width=0.49\linewidth, height=0.35\linewidth]{taxes_workers}
\includegraphics[width=0.49\linewidth, height=0.35\linewidth]{taxes_workers_pe1.0}
\caption{Savings taxes with private equity and tight collateral constraints ($\overline{\iota} = 1.0$)}
\label{tax_pe1.0} 
\end{figure} 

\begin{figure}[H]
\centering
\includegraphics[width=0.49\linewidth, height=0.35\linewidth]{taxes_entrepreneurs_pe}
%\includegraphics[width=0.49\linewidth, height=0.35\linewidth]{taxes_workers}
\includegraphics[width=0.49\linewidth, height=0.35\linewidth]{taxes_workers_pe}
\caption{Savings taxes with private equity and relaxed collateral constraints ($\overline{\iota} = 0.5$)}
\label{tax_pe} 
\end{figure} 

\begin{figure}[H]
\centering
\includegraphics[width=0.49\linewidth, height=0.35\linewidth]{r_pe}
\includegraphics[width=0.49\linewidth, height=0.35\linewidth]{r_pe1.0}
\caption{Interest rates with private equity ($\overline{\iota} = 0.5$ and $\overline{\iota} = 1.0$)}
\label{r_pe} 
\end{figure}

Relative to the benchmark market structure of Section \ref{DecenGenEq}, the presence of equity markets reduces the exposure of the entrepreneurs to the risk in their business. This increases the real interest rate from $\hat{\Pi} - \sqrt{\rho}\sigma \hat{x}$ to $\hat{\Pi} - \sqrt{\rho}\phi\sigma \hat{x}$ and therefore increases the savings taxes on all agents. Interestingly, in contrast with the benchmark decentralization from Section \ref{DecenGenEq}, the interest rate (and hence the tax on workers) now appears to be monotonic in agency frictions. In contrast, the tax on entrepreneurs' savings in Figure \ref{tax_pe1.0} no longer appears to be monotonic when collateral constraints are tight. 


\end{document}

\begin{figure}[H]
\centering
\includegraphics[width=0.49\linewidth, height=0.35\linewidth]{excess}
%\includegraphics[width=0.49\linewidth, height=0.35\linewidth]{taxes_workers}
\includegraphics[width=0.49\linewidth, height=0.35\linewidth]{excess_pe}
\caption{Excess return (risk premia)}
\label{excess} 
\end{figure} 


Finally, Figure \ref{leverage} depicts $\hat{k}$, the investment of each entrepreneur as a fraction of wealth, which is independent of market structure. 

\begin{figure}[H]
\centering
\includegraphics[width=0.49\linewidth, height=0.35\linewidth]{leverage}
\includegraphics[width=0.49\linewidth, height=0.35\linewidth]{leverage1.0}
%\includegraphics[width=0.49\linewidth, height=0.35\linewidth]{nu_B_minus_n_K}
\caption{Leverage of entrepreneurs in efficient allocations}
\label{leverage} 
\end{figure} 


\iffalse
\subsection{Private equity}\label{private_equity}

First suppose that the government allows agents to borrow and lend at $r = \rho_S$. For both of the following decentralizations, write $\hat{x} = x(\hat{S}, \overline{\omega})$ for $\hat{S}$ given in equation \eqref{statRCpsi} in Assumption \ref{existSTAT}. 

\begin{prop} \label{zerodriftEQrhoS}
When the no-absconding constraint does not hold with equality and the government issues a bond with price $r = \rho_S$, the stationary efficient allocation can be implemented as a stationary competitive equilibrium with linear taxes on savings and profits. The tax on worker savings is zero, the tax on profits is $\tau_{\Pi} = 1 - \phi$, and the tax on entrepreneurs' savings are $\tau_{sE} = \hat{x}^2$. The endowed total wealth of each entrepreneur as a fraction of aggregate capital is
\begin{align*}
\kappa_E & = \frac{\sqrt{\rho}\phi \sigma}{\rho \hat{x}(1-\psi)}
\end{align*}
while $\kappa_W = \kappa_Ee^{-\hat{x}^2/2}$, and the constant in the collateral constraint is given by $\hat{\omega} = \iota^{-1}e^{ - \hat{x}^2/2}$.
\end{prop}

\subsection{Government bond}\label{bond}

First suppose that the government allows agents to borrow and lend at $r = \rho_S$. For both of the following decentralizations, write $\hat{x} = x(\hat{S}, \overline{\omega})$ for $\hat{S}$ given in equation \eqref{statRCpsi} in Assumption \ref{existSTAT}. 

\begin{prop} \label{zerodriftEQrhoS}
When the no-absconding constraint does not hold with equality and the government issues a bond with price $r = \rho_S$, the stationary efficient allocation can be implemented as a stationary competitive equilibrium with linear taxes on savings and profits. The tax on worker savings is zero, the tax on profits is $\tau_{\Pi} = 1 - \phi$, and the tax on entrepreneurs' savings are $\tau_{sE} = \hat{x}^2$. The endowed total wealth of each entrepreneur as a fraction of aggregate capital is
\begin{align*}
\kappa_E & = \frac{\sqrt{\rho}\phi \sigma}{\rho \hat{x}(1-\psi)}
\end{align*}
while $\kappa_W = \kappa_Ee^{-\hat{x}^2/2}$, and the constant in the collateral constraint is given by $\hat{\omega} = \iota^{-1}e^{ - \hat{x}^2/2}$.
\end{prop}
I will not strive for complete generality, as the purpose of this section is mainly to understand how the results of the main text change under two extensions:
\begin{enumerate}
\item there is small number of more productive entrepreneurs subject to the same agency frictions as those in the benchmark case; and
\item one type of entrepreneur operates a risk-free technology (which may be interpreted as a corporate, or nonentrepreneurial sector).
\end{enumerate}
\fi

